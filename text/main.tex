%% Copernicus Publications Manuscript Preparation Template for LaTeX Submissions
%% ---------------------------------
%% This template should be used for copernicus.cls
%% The class file and some style files are bundled in the Copernicus Latex Package, which can be downloaded from the different journal webpages.
%% For further assistance please contact Copernicus Publications at: production@copernicus.org
%% https://publications.copernicus.org/for_authors/manuscript_preparation.html


%% Please use the following documentclass and journal abbreviations for preprints and final revised papers.

%% 2-column papers and preprints
\documentclass[gmd, manuscript]{copernicus}



%% Journal abbreviations (please use the same for preprints and final revised papers)


% Advances in Geosciences (adgeo)
% Advances in Radio Science (ars)
% Advances in Science and Research (asr)
% Advances in Statistical Climatology, Meteorology and Oceanography (ascmo)
% Annales Geophysicae (angeo)
% Archives Animal Breeding (aab)
% ASTRA Proceedings (ap)
% Atmospheric Chemistry and Physics (acp)
% Atmospheric Measurement Techniques (amt)
% Biogeosciences (bg)
% Climate of the Past (cp)
% DEUQUA Special Publications (deuquasp)
% Drinking Water Engineering and Science (dwes)
% Earth Surface Dynamics (esurf)
% Earth System Dynamics (esd)
% Earth System Science Data (essd)
% E&G Quaternary Science Journal (egqsj)
% European Journal of Mineralogy (ejm)
% Fossil Record (fr)
% Geochronology (gchron)
% Geographica Helvetica (gh)
% Geoscience Communication (gc)
% Geoscientific Instrumentation, Methods and Data Systems (gi)
% Geoscientific Model Development (gmd)
% History of Geo- and Space Sciences (hgss)
% Hydrology and Earth System Sciences (hess)
% Journal of Bone and Joint Infection (jbji)
% Journal of Micropalaeontology (jm)
% Journal of Sensors and Sensor Systems (jsss)
% Magnetic Resonance (mr)
% Mechanical Sciences (ms)
% Natural Hazards and Earth System Sciences (nhess)
% Nonlinear Processes in Geophysics (npg)
% Ocean Science (os)
% Primate Biology (pb)
% Proceedings of the International Association of Hydrological Sciences (piahs)
% Scientific Drilling (sd)
% SOIL (soil)
% Solid Earth (se)
% The Cryosphere (tc)
% Weather and Climate Dynamics (wcd)
% Web Ecology (we)
% Wind Energy Science (wes)


%% \usepackage commands included in the copernicus.cls:
%\usepackage[german, english]{babel}
%\usepackage{tabularx}
%\usepackage{cancel}
%\usepackage{multirow}
%\usepackage{supertabular}
%\usepackage{algorithmic}
%\usepackage{algorithm}
%\usepackage{amsthm}
%\usepackage{float}
%\usepackage{subfig}
%\usepackage{rotating}

% Shortcuts for math in methods
\newcommand{\up}{\uparrow}
\newcommand{\down}{\downarrow}
\newcommand{\sky}{\mathrm{sky}}
\newcommand{\ground}{\mathrm{ground}}
\newcommand{\leaf}{\mathrm{leaf}}
\newcommand{\wood}{\mathrm{wood}}
\newcommand{\plant}{\mathrm{plant}}
\newcommand{\soil}{\mathrm{soil}}
\newcommand{\site}{\mathrm{site}}
\newcommand{\LAI}{\mathrm{LAI}}
\newcommand{\WAI}{\mathrm{WAI}}
\newcommand{\TAI}{\mathrm{TAI}}
\newcommand{\SLA}{\mathrm{SLA}}
\newcommand{\DBH}{\mathrm{DBH}}
\newcommand{\EDR}{\mathrm{EDR}}
\newcommand{\bbbl}{\mathrm{b1Bl}}
\newcommand{\bebl}{\mathrm{b2Bl}}
\newcommand{\bbbw}{\mathrm{b1Bw}}
\newcommand{\bebw}{\mathrm{b2Bw}}
\newcommand{\mat}[1]{\mathbf{#1}}

\begin{document}

\title{Cutting out the middleman: Calibrating and validating a dynamic vegetation model (ED2-PROSPECT5) using remotely sensed surface reflectance}


% \Author[affil]{given_name}{surname}

\Author[1]{Alexey N.}{Shiklomanov}
\Author[2]{Michael C.}{Dietze}
\Author[3]{Istem}{Fer}
\Author[3]{Toni}{Viskari}
\Author[4]{Shawn P.}{Serbin}

\affil[1]{NASA Goddard Space Flight Center, Greenbelt, MD, USA}
\affil[2]{Department of Earth and Environment, Boston University, Boston, MA, USA}
\affil[3]{Finnish Meteorological Institute, Helsinki, Finland}
\affil[4]{Environmental and Climate Sciences Department, Brookhaven National Laboratory, Upton, NY, USA}

%% The [] brackets identify the author with the corresponding affiliation. 1, 2, 3, etc. should be inserted.

%% If an author is deceased, please mark the respective author name(s) with a dagger, e.g. "\Author[2,$\dag$]{Anton}{Aman}", and add a further "\affil[$\dag$]{deceased, 1 July 2019}".

%% If authors contributed equally, please mark the respective author names with an asterisk, e.g. "\Author[2,*]{Anton}{Aman}" and "\Author[3,*]{Bradley}{Bman}" and add a further affiliation: "\affil[*]{These authors contributed equally to this work.}".


\correspondence{Dr.\ Alexey N. Shiklomanov (alexey.shiklomanov@nasa.gov)}

\runningtitle{Calibrating a DGVM using reflectance}

\runningauthor{Shiklomanov \emph{et al.}}





\received{}
\pubdiscuss{} %% only important for two-stage journals
\revised{}
\accepted{}
\published{}

%% These dates will be inserted by Copernicus Publications during the typesetting process.


\firstpage{1}

\maketitle



\begin{abstract}
  Canopy radiative transfer is the primary mechanism by which models relate vegetation composition and state to the surface energy balance, which is important to light- and temperature-sensitive plant processes as well as understanding land-atmosphere feedbacks.
  In addition, certain parameters (e.g., SLA) that have an outsized influence on vegetation model behavior can be constrained by observations of shortwave reflectance, thus reducing model predictive uncertainty.
  Importantly, calibrating against radiative transfer outputs allows models to directly use remote sensing reflectance products without relying on highly derived products (such as MODIS leaf area index) whose assumptions may be incompatible with the target vegetation model and whose uncertainties are usually not well quantified.
  Here, we coupled the two-stream representation of canopy radiative transfer in the Ecosystem Demography model (ED2) with a leaf radiative transfer model (PROSPECT 5) and a simple soil reflectance model to predict full-range, high spectral resolution surface reflectance that is dependent on the underlying ED2 model state.
  We then calibrated this model against estimates of hemispherical reflectance (corrected for directional effects) from the NASA Airborne VIsible/InfraRed Imaging Spectrometer (AVIRIS) and survey data from 54 temperate forest plots in the northeastern United States.
  The calibration significantly reduced uncertainty in model parameters related to leaf biochemistry and morphology and canopy structure for five plant functional types.
  Using a single common set of parameters across all sites, the calibrated model was able to accurately reproduce surface reflectance for sites with highly varied forest composition and structure.
  However, the calibrated model's predictions of leaf area index (LAI) were less robust, capturing only 46\% of the variability in the observations.
  Comparing the ED2 radiative transfer model with a similar model commonly used in remote sensing studies (PRO4SAIL) illustrated structural errors in the ED2 representation of direct radiation backscatter that resulted in systematic underestimation of reflectance.
  In addition, we also highlight that, to directly compare with a two-stream radiative transfer model like EDR, we had to perform an additional processing step to convert the directional reflectance estimates of AVIRIS to hemispherical reflectance (a.k.a., ``albedo'').
  In future work, we recommend that vegetation models add the capability to predict directional reflectance, to allow them to more directly assimilate a wide range of airborne and satellite reflectance products.
  We ultimately conclude that despite these challenges, using dynamic vegetation models to predict surface reflectance is a promising avenue for model calibration and validation using remote sensing data.
\end{abstract}


\copyrightstatement{TEXT}


\section{Introduction}

My previous chapters have shown that models have much to gain, both in terms of direct parameter constraint from trait observations and from new process representations that emerge from trait ecology more broadly.
However, there are limits on the extent to which traits alone can improve models.
For one, even after examining a broad range of inter- and intraspecific factors, large fractions of variability in plant function remain unexplained.
Moreover, vegetation models are simplified abstractions of reality, with many processes omitted or represented by simplistic empirical equations with little-to-no physical basis and therefore no directly measurable trait that can serve as a parameter constraint.
In these cases, models can only be calibrated via their emergent predictions of state variables.

Many previous efforts have used various data streams calibrate or constrain dynamic vegetation model parameters and states.
Among these data streams, remote sensing is particularly promising due to its consistent measurement methodology and largely uninterrupted global coverage.
Data products derived from remote sensing observations have been effectively used to constrain, among others,
phenology~\parencite{knorr2010carbon, viskari2015modeldata},
absorbed photosynthetically-active radiation~\parencite{peylin2016new, schurmann2016constraining},
and primary productivity~\parencite{macbean2018strong}.

However, there are issues with using derived remote sensing products to calibrate ecosystem models.
Relationships of surface reflectance variables (such as vegetation indices) with characteristics of vegetation structure and function estimated by models are complex.
For example, the assumption of a simple linear relationship between the normalized difference vegetation index and absorbed photosynthetically-active radiation (e.g.\ Peylin et al.~2016) \nocite{peylin2016new} has long been shown to be sensitive to variability in soil and leaf optical properties~\parencite{myneni1994relationship}, and is known to vary across spatial scales and sensor configurations~\parencite{fensholt2004evaluation}.
A related issue is that subtle but significant differences in the ways vegetation variables are defined, by both models and data products, can significantly affect the interpretation of remotely sensed data~\parencite{carlson1997relation}. % TODO: Needs an example
Furthermore, uncertainties in derived remote sensing data products are often poorly quantified but known to be significant,
to the extent that some studies advise against working with individual pixel values in favor of averaging across adjacent pixels (thereby dramatically reducing the spatial resolution) to achieve reasonable accuracy~\parencite{wenzeyang2006modis, wang2004evaluation}.
Although these issues could be partially alleviated by robust, pixel-level uncertainty estimates for remote sensing data products, such estimates are generally not widely available for most data products.
Collectively, these issues, combined with differences in sensor configuration and design, result in large differences in estimates of surface characteristics across different remote sensing instruments that lead directly to different estimates of carbon storage and flux~\parencite{liu_2018_satellite}. % TODO: Lots of differences

One way to overcome the limitations of derived remote sensing data products while still leveraging the capabilities of remote sensing is to work directly with the observed surface reflectance.
In the context of dynamic vegetation modeling, this can be accomplished by coupling these models with leaf and canopy radiative transfer models that simulate surface reflectance as a function of known surface characteristics~\parencite{quaife2008assimilating}.
Such an approach takes advantage of the fact that surface reflectance contains valuable information about vegetation structure and function
without relying on the independent retrieval of these characteristics from reflectance data alone, which is often an ill-posed problem~\parencite{combal2003retrieval, lewis2007spectral}. % TODO: Dietze -- needs more explanation
Moreover, besides enabling assimilation of remotely sensed data, training models to accurately simulate surface reflectance is essential to properly quantifying and testing hypotheses related to vegetation-climate interactions and feedbacks.
% TODO: Dietze -- (1) This should start a new paragraph. (2) Mention that the other studies (e.g. Quaife 2008) used an external RTM, not the model's own. This is novel in this work. (3) Mention Toni's RTM uncertainty analysis in this paragraph?
For instance, the net climate effect of ongoing changes in Arctic vegetation composition depends on the balance of opposing radiative (lower albedo) and latent (increased transpiration) energy feedbacks~\parencite{swann2010changes}, so forecasting this effect requires accurate models of canopy energy transfer.
More fundamentally, light availability is a key control of photosynthesis and therefore has
immediate, direct consequences for individual plant function~\parencite{hikosaka1995model, robakowski_2004_growth, niinemets2016withincanopy, keenan2016global}
as well as longer-term, indirect consequences for competition and ecological succession~\parencite{niinemets2006tolerance, kitajima2013leaf, falster2017multitrait}.

Recognition of the importance of these processes has led to the development of vegetation models with explicit representations of canopy radiative transfer.
The most accurate canopy radiative transfer models capture both vertical and horizontal heterogeneity with very high spatial resolution~\parencite{widlowski2007third}.
However, such models are usually too computationally intensive for dynamic vegetation models, which employ various approximations based on simplifying assumptions to make the problem more tractable~\parencite{fisher2017vegetation}.
One common approach is the ``two-stream approximation'', which simplifies the problem of directional scattering within a medium by modeling the hemispherical integral of fluxes rather than individual, directional components.
In the context of radiative transfer in plant canopies, many different two-stream formulations have been developed, of which I highlight two:
One formulation was developed by Kubelka and Munk (1931)\nocite{kubelka_munk_1931} and later adapted to vegetation canopies by Allen, Gayle, and Richardson (1970)\nocite{allen_1970_plant} and further refined by Suits (1971)\nocite{suits_1971_calculation}, Verhoef (1985)\nocite{verhoef_1984_sail}, and others.
This theory forms the foundation of the SAIL canopy radiative transfer model~\parencite{verhoef_1984_sail} and its derivatives~\parencite[e.g. 4SAIL][]{verhoef_2007_4sail}, which have been used extensively in the remote sensing community for modeling and retrieving vegetation characteristics from spectral data~\cite{jacquemoud_2009_prosail}.
Another was developed by Meador and Weaver (1980)\nocite{meador_1980_twostream} for atmospheric radiative transfer, and was subsequently adapted to canopy radiative transfer by Dickinson (1983)\nocite{dickinson_1983_land} and refined by Sellers (1985)\nocite{SELLERS_1985_canopy}.
%Key assumptions of this approach are that all diffuse radiative fluxes are isotropic (i.e.\ scattering is equal in all directions) and that all canopy elements are sufficiently far from each other that there is no inter-particle shading~\parencite{meador_1980_twostream,dickinson_1983_land,SELLERS_1985_canopy}.
Due to its theoretical simplicity and low computational demand, this is the approach commonly used to represent radiative transfer in ecosystem models, including the Community Land Model~\parencite[CLM,][]{clm45_note} and the Ecosystem Demography model~\cite[ED,][]{Moorcroft_2001_ED,Medvigy_2009_ED2}.
The version of this scheme used in ED2 (and derivative models) is fairly unique in its explicit representation of multiple canopy layers, which allows ED2 to simulate competition for light, a key component of modeling vegetation demographics~\parencite{fisher_2017_vegetation}.
However, compared to physiological processes, the structure and parameterization of canopy radiative transfer schemes in demographic models has received relatively little attention.
When canopy radiative transfer has been considered, it was shown to be important to a wide range of physiological and demographic processes.
For example, using a modified version of the ED model, Fisher et al. (2010)\nocite{fisher_2010_assessing} showed that excessive light absorption by the top cohort resulted in unrealistically excessive growth of canopy trees at the expense of understory trees.
Similarly, an analysis by Viskari et al.\ (in revision) \nocite{Viskari_inreview_ED} demonstrated that the Ecosystem Demography (ED2) model's predictions of ecosystem energy budget, productivity, and composition are highly sensitive to the parameterization of the model's representation of canopy radiative transfer.
Understanding and improving representations of canopy radiative transfer in dynamic vegetation models is therefore critical to accurate projections of the fate of the terrestrial biosphere.

Building on the work of Viskari et al., the objective of this chapter is to develop and demonstrate the calibration and validation of the ED2 model using remotely sensed surface reflectance.
First, I link the canopy radiative transfer model in ED2 with the PROSPECT leaf radiative transfer model to allow ED to predict full-range, hyperspectral surface reflectance at each time step.
Second, I calibrate this coupled leaf-canopy radiative transfer model at a number of sites in the US Midwest and Northeast where coincident plot vegetation survey data and observations of the NASA Airborne Visible/InfraRed Imaging Spectrometer (AVIRIS) are available.

\section{Methods}

\subsection{Model description}

The Ecosystem Demography version 2.2 (ED2) model simulates plot-level vegetation dynamics and biogeochemistry~\parencite{moorcroft_2001_method, medvigy2009mechanistic, longo_2019_ed1}.
By grouping individuals of similar size, structure, and composition together into cohorts, ED2 is capable of modeling patch-level competition in a computationally efficient manner.
Relevant to this work, ED2 includes a multi-layer canopy radiative transfer model that is a generalization of the two-layer two-stream radiative transfer scheme in CLM 4.5~\parencite{clm45_note}, which in turn is derived from \textcite{sellers1985canopy}.
A complete description of the model derivation is provided in the supplementary information of \textcite{longo_2019_ed1}, but for completeness, we provide an abbreviated description below:

Let $n$ be the number of cohorts in a patch, with $1$ being the shortest and $n$ being the tallest.
The ED2 radiative transfer model calculates the total hemispherical upward and downward flux at each cohort, as well as at the ground ($0$) and the interface above the canopy ($n+1$).
The ED2 radiative transfer model solves for the upward and downward flux at $n+2$ levels in the canopy
The ED2 radiative transfer model solves for the overall hemispherical (i.e.\ diffuse) surface albedo
in two spectral ``bands''---visible and near-infrared---as a function of
each cohort's leaf and wood area indices and
PFT-specific parameters for leaf and wood reflectance and transmittance, canopy clumping factor, and leaf orientation factor.

The direct radiation flux is modeled as an exponential attenuation curve through the canopy based on each layer's transmissivity ($\tau_r$),
which in turn is a function of the total area index ($TAI$) of the canopy layer and the inverse optical depth ($\mu_r$):

\begin{equation}\label{eq:tau_r}
  \tau_r = e ^ {- \frac{TAI}{\mu_r}}
\end{equation}

The total area index ($TAI$) is the sum of the wood area index ($WAI$) and the effective leaf area index, with the latter calculated as the product of the true leaf area index ($LAI$) and the clumping factor ($c$, defined on the interval $(0, 1)$ where 0 is a ``black hole''---all leaf mass concentrated in a single point---and 1 is a homogenous closed canopy):

\begin{equation}\label{eq:tai}
  TAI = c LAI + WAI
\end{equation}

% TODO: Explain where WAI comes from, especially since it's relevant to the analysis.

The true leaf area index for each PFT is calcluated in two stages:
First, the total leaf biomass is calculated from the diameter at breast height via an exponential allometric equation parameterized for each PFT\@.
Second, the leaf biomass is converted to leaf area index through the PFT-specific specific leaf area (SLA).

The optical depth is calculated based on the projected area ($p$) and the solar zenith angle ($\theta$):

\begin{equation}
  \mu_r = \frac{\cos{\theta}}{p}
\end{equation}

The projected area ($p$) is a function of the leaf orientation factor ($f$):

\begin{equation}\label{eq:phi1}
  \phi_1 = 0.5 - f (0.633 + 0.33 f) \\
\end{equation}

\begin{equation}\label{eq:phi2}
  \phi_2 = 0.877 (1 - 2 \phi_1) \\
\end{equation}

\begin{equation}
  p = \phi_1 + \phi_2 \cos{\theta} 
\end{equation}

% TODO: Explain where all of these constants come from.

The diffuse radiation flux is more complicated because light is scattered internally within canopy layers.
Unlike the Community Land Model, which solves only for sunlit and shaded leaves,
ED2 calculates the full canopy radiation profile by parameterizing the two-stream equations for each layer (as well as soil and atmosphere boundary conditions)
and then using a linear system solver to solve for the radiation profile.
For each layer, leaf and wood forward scattering ($\omega_+$) are just the sums of their respective reflectance ($r$) and transmittance ($t$) values:

\begin{equation}
   \omega_+ = r + t 
\end{equation}

Leaf and wood backscatter ($\omega_-$) are a function of their respective reflectance and transmittance values as well as the leaf orientation factor ($f$):

\begin{equation}\label{eq:backscatter_leaf}
   \omega_- = \frac{r + t + 0.25 (r - t) {(1 + f)} ^ 2}{2 (r + t)} 
\end{equation}

% TODO: More explanation about these equations. 

Overall scatter ($\iota$) and backscatter ($\beta$) of all elements in a canopy layer is modeled as the average of leaf and wood scatter, weighted by their respective area indices:

\begin{equation}\label{eq:wl}
  w_l = \frac{LAI}{LAI + WAI}
\end{equation}

\begin{equation}\label{eq:ww}
  w_w = \frac{WAI}{LAI + WAI}
\end{equation}

\begin{equation}\label{eq:scatter}
  \iota = w_l \omega_{+,l} + w_w \omega_{+,w}
\end{equation}

\begin{equation}\label{eq:backscatter}
  \beta = w_l \omega_{-,l} + w_w \omega_{-,w}
\end{equation}

The inverse optical depth for diffuse radiation ($\mu_f$) is calculated from the coefficients $\phi_1$ and $\phi_2$ (see equations~\ref{eq:phi1} and~\ref{eq:phi2}):

\begin{equation}
   \mu_f = \frac{1 - \phi_1 \ln{(1 + \frac{\phi_2}{\phi_1 \phi_2})}}{\phi_2} 
\end{equation}

Note that $\mu_f$ simplifies to 1 when orientation factor is 0 (random, spherical distribution of leaf angles).
Collectively, these coefficients are used to calculate the optical depth for diffuse radiation ($\tau_f$):

\begin{equation}
  \epsilon = 1 - 2\beta
\end{equation}

\begin{equation}
  \lambda = \frac{\sqrt{(1 - \epsilon\iota) (1 - \iota)}}{\mu_f}
\end{equation}

\begin{equation}
  \tau_f = e ^ {\lambda TAI}
\end{equation}

The remaining coefficients are described in the Community Land Model manual~\parencite{clm45_note}.
% Should probably finish this as an appendix or something at some point.
% TODO: Mike wants this fleshed out more -- comment "What coefficients?"

By default, ED takes as parameters PFT-specific leaf and wood reflectance and transmittance values with one value each for the visible and near-infrared spectral regions. 
For this analysis, I first modified ED to take an arbitrary number of leaf and wood reflectance transmittance values.
From there on, I simulated leaf reflectance and transmittance using the PROSPECT 5 leaf RTM (see Chapters 2 and 3).
For soil and wood reflectance, I used means of the corresponding spectra from Asner (1998), resampled to 1 nm resolution. \nocite{asner_1998_biophysical}
The final coupled PROSPECT-ED canopy radiative transfer model (hereafter known as ``EDR'') has 10 parameters for each PFT\@:
5 parameters for PROSPECT (number of mesophyll layers, and area-based chlorophyll, carotenoid, water, and dry matter contents),
specific leaf area,
base and exponent for the leaf allometry,
and clumping and orientation factors.

\subsection{Sensitivity analysis}

To provide a basis for understanding the behavior of EDR, I performed a one-at-a-time sensitivity analysis to explore how its reflectance predictions vary with each leaf optical and canopy structural parameter.
To assess the mathematical foundation of the EDR canopy model (i.e.\ the Sellers two-stream scheme) without the confounding influence of multiple cohorts, I first performed this sensitivity analysis on a simulated plot containing only a single mature tree cohort.
For comparison, I also included simulations using the 4SAIL canopy radiative transfer model.
The 4SAIL model simulates four reflectance ``streams''---diffuse (hemispherical) and direct (directional) reflectance for both diffuse and direct incident radiation---\
but because EDR only simulates diffuse reflectance and its input is dominated by direct radiation (at least on sunny days, which are necessary for satellite and high-altitude airborne data), I used the ``directional-hemispherical'' output for all comparisons.
For its representation of leaf angle distribution, 4SAIL uses an ellipsoidal model that takes as input the mean leaf inclination angle ($\theta$), which is related to EDR's leaf orientation factor ($f$) by:

\begin{equation}\label{eq:orient_lidf}
  \cos \theta = \frac{1 + f}{2}
\end{equation}

Unlike EDR, 4SAIL does not account for canopy clumping.
(4SAIL does have a ``hot spot'' parameter to account for strong bi-directional reflectance effects from structurally heterogeneous canopies, but this parameter only affects the bi-directional reflectance, which was not used in this analysis).

To investigate the way EDR models interactions between canopy layers, I also performed a similar sensitivity analysis on a simulated plot with two cohorts---a dominant early successional cohort and a sub-dominant mid-successional cohort.
I examined the sensitivity of total canopy reflectance to the optical properties of both the dominant and sub-dominant cohort, and looked at how this sensitivity was affected by clumping in the upper cohort.

\subsection{Model calibration}

For model calibration, I selected 47 sites from the NASA Forest Functional Types (FFT) field campaign that contained plot-level inventory data (stem density, species identity, and DBH) coincident with observations of the NASA Airborne Visible/Infrared Imaging Spectrometer (AVIRIS).
These sites are mostly located in the United States Upper Midwest with several sites also in upstate New York and western Maryland, and include stands dominated by either evergreen or deciduous trees and spanning a wide range of structures, from dense groups of saplings (bottom right) to sparse groups of large trees (top left) (Figure~\ref{fig:sites}).
Based on ED's PFT definitions, these sites contained a total of five different temperate plant functional types: Early successional hardwood, northern mid-successional hardwood, late successional hardwood, northern pine, and late successional conifer.

% \begin{figure}
%   \centering
%   \includegraphics[width=\textwidth]{figures/sites_both.pdf}
%   \caption{\
%     Sites selected for analysis, in ``stand structure'' (\textit{main figure}) and geographic (\textit{inset}) space.
%     Colors indicate the fraction of the stand that is made up of evergreen PFTs.
%     %Large points with labels indicate sites that were selected for forward simulations.
%     % TODO: Dietze -- (1) Axes are reversed relative to Andrews ESM paper. (2) Stand density values are way too low. Basically, check against Andrews ESM paper and try to match units.
%   }\label{fig:sites}
% \end{figure}
% TODO: Change colors to be top cohort

I calibrated EDR using the same general Bayesian inversion as in Chapter 3.
The inversion fit all sites simultaneously, such that at every MCMC iteration, the algorithm proposed a set of all parameter values for each PFT and simulated spectra for each site based on its observed composition and structure.
Because of unrealistic values in the shortwave infrared spectral region in the AVIRIS observations, likely caused by faulty atmospheric correction, I only calibrated the model with observations from 400 to 1300 nm.
In addition, I changed the fixed variance model used in Chapters 2 and 3 to a two-parameter heteroskedastic variance model ($\sigma = a + bX$) to account for the fact that both model and observation errors are typically proportional to reflectance values.
To generate the initial history state files required by EDR, I ran ED2 itself for one day in midsummer (July 1), starting from vegetation initial conditions based on observed composition and structure.

For priors on the five PROSPECT parameters and specific leaf area, I performed a hierarchical multivariate analysis (see Chapter 1) on PROSPECT parameters estimated from chapter 3 and, where available, direct measurements of specific leaf area. 
For priors on the leaf biomass allometry parameters, I fit a multivariate normal distribution to allometry coefficients from Jenkins et al.~(2003, 2004) using the \texttt{PEcAn.allometry} package. \nocite{jenkins_2003_allom,jenkins_2004_allom} 
For the clumping factor, I used a uniform prior across its full range (0 to 1), and for the leaf orientation factor, I used a weakly informative re-scaled beta distribution centered on 0.5.

To alleviate issues with strong collinearity between the two allometry coefficients and the specific leaf area, I decided to remove the allometry exponent coefficient (but not the intercept) from the calibration by fixing it at its prior mean for each plant functional type.
Doing so dramatically improved the stability of the inversion algorithm and the accuracy of the results.

I evaluated the performance of the calibrated model by comparing the posterior credible intervals of modeled spectra against the AVIRIS observations at each site.
To assess the role of model structure in predictive error, I also included predictions using the 4SAIL model parameterized with the posterior means from the EDR calibration (except for clumping factor, which is absent from 4SAIL).
In addition, I compared model predictions of leaf area index (which depend on parameters calibrated in the model) against field observations.

%%% Local Variables:
%%% mode: latex
%%% TeX-master: "../dissertation"
%%% End:

\section{Results}

\subsection{Model calibration}

\begin{figure}
  \centering
  \includegraphics[width=\textwidth]{figures/posterior-pft}
  \caption{\label{fig:posterior-pft}\
    Marginal prior (pre-calibration; grey) and posterior (post-calibration; black) distributions of PFT-specific parameters
    related to leaf biochemistry and canopy structure.
    Distributions are shown as violin plots (rotated and mirrored kernel density plots).
  }
\end{figure}

Model calibration improved the precision of most PFT-specific parameter estimates, including parameters whose prior distributions were informative (Figure~\ref{fig:posterior-pft}).
For leaf traits, PFT rankings of the posterior estimates largely followed the relative positions of the priors.
The effective number of leaf layers (PROSPECT \emph{N} parameter) was higher for needleleaved than broadleaved PFTs, with the highest value for Northern Pine and the lowest value for Mid Hardwood.
Estimated total chlorophyll content (\emph{Cab}) were similar across all PFTs, with the highest values for Early and Mid Hardwood followed closely by Late Hardwood and Late Conifer, and the lowest values for Northern Pine.
Estimates of leaf total carotenoid (\emph{Car}), water (\emph{Cw}), and dry matter contents (\emph{Cm}) had distributions that overlapped for all PFTs, though the central tendency of Late Conifer was slightly higher than other PFTs for all three traits.
Finally, estimated specific leaf area (SLA) was highest in Early Hardwood, followed by Late Hardwood and Mid Hardwood, and was comparably low for Northern Pine and Late Conifer.

Compared to leaf traits, canopy structural traits had less informative (and PFT-agnostic) priors and were more constrained by the calibration.
In some cases, this constraint suggested differences between PFTs, though most estimated parameter distributions were still mutually overlapping.
For example, leaf orientation factors and, to a lesser extent, canopy clumping factors and leaf biomass allometry parameters (\emph{b1Bl}) were higher for Mid- and Late-successional broadleaved PFTs than other PFTs.
Meanwhile, Northern Pine had the lowest leaf biomass allometry parameters and clumping and orient factors, and the highest wood biomass allometry parameter (\emph{b1Bw}).
Calibration provided only limited constraint on site-specific soil optical properties, with posterior estimates that were typically almost as wide as the uninformative prior distributions for all but a few specific sites (Figure~\ref{fig:posterior-soil}).

\begin{figure}
  \centering
  \includegraphics[width=\textwidth]{figures/spec-error-all}
  \caption{\label{fig:spec-error-all}\
    Comparison between AVIRIS observed (black),
    EDR predicted (mean prediction in green, 95\% posterior predictive interval in gray),
    and PRO4SAIL predicted (red)
    surface reflectance for each site used in the calibration.
    Sites are sorted in order of decreasing mean difference between observed and EDR predicted reflectance
    (largest underestimates first, largest overestimates last).
  }
\end{figure}

\begin{figure}
  \centering
  \includegraphics[width=\textwidth]{figures/spec-error-aggregate}
  \caption{\label{fig:spec-error-aggregate}\
    Difference between AVIRIS observed and EDR predicted (mean) site surface reflectance.
    One line per site and observation is shown (some sites had multiple observations).
  }
\end{figure}

The ability of EDR to reproduce observed spectra at every site was site-dependent (Figures~\ref{fig:spec-error-all} and~\ref{fig:spec-error-aggregate}).
The largest differences between observed and mean predicted reflectance were in the near-infrared region, particularly from 775 to 1100 nm,
while predictions in the visible range agreed well with observations in all but a few cases.
That said, the EDR predictive interval overlapped observations in all but a few individual cases, suggesting that our estimates of model uncertainty are realistic.
Predictions from the EDR model were closer to observations than predictions by the PRO4SAIL model across most sites, with the latter significantly overestimating reflectance in all but a few specific cases.

\begin{figure}[ht]
  \centering
  \includegraphics[width=\textwidth]{figures/late-conifer-sites}
  \caption{\label{fig:late-conifer-sites}\
    Predicted and observed surface reflectance, and histogram of observed diameter at breast height (DBH) census observations,
    for sites with large Late Conifer presence.
  }
\end{figure}

Mismatch between observed and predicted reflectance was primarily explained by two factors.
First, EDR predictions of surface reflectance were more closely related to stand maturity (represented by mean DBH) at each site than in the observations (Figure~\ref{fig:ndvi-dbh}).
Second, EDR predictions were particularly inaccurate at sites where Late Conifer was the dominant PFT (Figure~\ref{fig:late-conifer-sites});
the inaccurate predicted spectra at these sites were likely driven by the calibration algorithm's really high estimate for wood allometry and therefore high wood area index (Figure~\ref{fig:posterior-pft}).

% At a majority of the sites, EDR systematically over-predicted reflectance in the visible range (Figure~\ref{fig:spec_error_vis}), while errors in the near-infrared region were more variable.
% As shown in the sensitivity analysis, this consistent over-prediction of visible reflectance is likely driven by wood reflectance (Figure~\ref{fig:wood_compare}).
% 4SAIL also showed a lot of site-to-site variability in its performance, but generally performed better in the visible range than EDR\@.

\begin{figure}
  \centering
  \includegraphics[width=\textwidth]{figures/lai-pred-obs}
  \caption{\
    Predictions of leaf area index by EDR, compared to observed values.
    % Colors indicate the plant functional type of the tallest cohort at each site.
    % TODO: Add by-PFT regression line
  }\label{fig:lai-pred-obs}
\end{figure}

The ability of EDR to reproduce observed leaf area index was also strongly site-dependent, with some of the accuracy explained by the functional type of the tallest cohort (Figure~\ref{fig:lai-pred-obs}).
In general, EDR tended to over-predict leaf area index for conifer-dominated stands and under-predict for hardwood-dominated stands.
For mid- and late-hardwood-dominated stands in particular, EDR predicted substantial variability in leaf area index that was not present in the observations.

% \begin{figure}
%   \centering
%   \includegraphics[width=\textwidth]{4_edr/figures/explore_spectra/ed_cumlai_plot.pdf}
%   \caption{%
%     Vertical profile of cumulative leaf area index and composition at each site in this analysis.
%     Vertical black lines indicate the mean $\pm$ 1 standard deviation of the observed leaf area index.
%     Sites are arranged in the same order as Figures~\ref{fig:spec_error_all} and~\ref{fig:spec_error_vis}.
%   }\label{fig:lai_profile}
% \end{figure}

Mismatch between EDR predictions and AVIRIS reflectance were likely caused by a number of factors related to site composition and structure (Figures~\ref{fig:spec_error_vis} and~\ref{fig:lai_profile}).
In some sites, the mismatch was most likely due to a mismatch in leaf area index, such as BI02, BI03, and MN06.
Notably, at BI02 and BI03 (and several other sites), EDR and 4SAIL show opposite biases---EDR over-predicts visible reflectance but successfully captures the near-infrared reflectance, while 4SAIL does the opposite (Figure~\ref{fig:spec_error_all}).
This can be linked to the two models' different responses to leaf area index revealed in the sensitivity analysis (Figure~\ref{fig:sensitivity_structure_single}).

Where late hardwood trees were relatively abundant near the top of the canopy (Figure~\ref{fig:lai_profile}), EDR often over-predicted reflectance in the red (sites BH03, BH05, BH10, and BI01; Figure~\ref{fig:spec_error_vis}) even though the LAI retrieval was reasonably accurate.
This was likely related to the low inversion estimate of late hardwood clumping factor (Figure~\ref{fig:posterior-pft}), which tends to emphasize the much redder wood and soil background (Figure~\ref{fig:sensitivity_structure_single}).
However, some other late hardwood-dominated sites showed good performance for both spectra and leaf area index, such as NC17, NC22, and OF04 (Figure~\ref{fig:spec_error_all}).
Similarly, the high clumping factor estimate for late conifer trees (Figure~\ref{fig:posterior-pft}) was compensated over-predicted leaf area index (driven by increases in the leaf biomass allometry coefficient; Figure~\ref{fig:lai-pred-obs}), as in sites OF01, SF01, and SF04.
More generally, EDR tended to perform best in mature stands comfortably dominated by early or mid hardwoods or northern pines.

\subsection{Sensitivity analysis}

% \begin{figure}
%   \centering
%   \includegraphics[width=\textwidth]{4_edr/figures/explore_spectra/edr_sensitivity_leaf_single.pdf}
%   \caption{\
%     Sensitivity of EDR and 4SAIL predicted canopy reflectance to leaf optical traits.
%     For all figures, leaf area index is fixed at 4.88.
%     EDR simulations are for a single-cohort canopy (Early Hardwood) with
%     clumping factor 0.09 and orientation factor 0.06.
%     4SAIL predictions are for directional-hemispherical reflectance.
%   }\label{fig:sensitivity_leaf_single}
% \end{figure}

The general character of the sensitivities of EDR and 4SAIL to leaf optical properties is similar,
but the magnitudes of these sensitivites are different (Figure~\ref{fig:sensitivity_leaf_single}).
EDR consistently shows significantly higher reflectance across most of the spectrum than 4SAIL\@.
Sensitivity to leaf mesophyll structure is lower in EDR than 4SAIL, while sensitivity for chlorophyll and water contents is comparable.
Sensitivity to leaf dry mass per area is comparable for both models in the shortwave infrared, but significantly higher for 4SAIL in the near infrared.

% \begin{figure}
%   \centering
%   \includegraphics[width=\textwidth]{4_edr/figures/explore_spectra/sensitivity_single_pft.pdf}
%   \caption{\
%     Sensitivity of EDR and 4SAIL predicted canopy reflectance to leaf area index (\textit{top}) and leaf orientation factor (\textit{middle}).
%     (\textit{Bottom}) Sensitivity of EDR predicted canopy reflectance and clumping factor,
%     with dashed line indicating 4SAIL predictions for the same leaf area index, PROSPECT parameters, and approximately equivalent soil background.
%     Configuration is the same as in Figure~\ref{fig:sensitivity_leaf_single}.
%   }\label{fig:sensitivity_structure_single}
% \end{figure}

% TODO: Add sensitivity to WAI and soil.

EDR and 4SAIL show different responses to leaf area index (Figure~\ref{fig:sensitivity_structure_single}).
Although both models predict declines in reflectance with increasing leaf area in the visible and shortwave infrared range,
4SAIL also predicts a decline in the near infrared while EDR predicts an increase.
Furthermore, 4SAIL predicts more reflectance sensitivity at low leaf area indices and saturation of reflectance around 4, while EDR shows a more gradual decline in sensitivity, particularly in the near-infrared range. 
An important caveat to these results is that, particularly at low leaf area index, they are strongly dependent on the value of the background soil reflectance.
To match the EDR default (see Methods), 4SAIL was configured with a relatively bright soil reflectance (i.e.\ a fairly dry soil), which explains the decline in near-infrared reflectance as leaf area increases.
When the soil background is dark, SAIL shows increasing near-infrared reflectance with increasing leaf area (but saturating to the same value as the contribution of the soil background becomes negligible at high leaf area).

Similarly to leaf area, EDR and 4SAIL agree on the directionality leaf orientation effects on reflectance (declining reflectance with increasingly vertical leaves), but differ in their sensitivities, with EDR having a much lower sensitivity to changing leaf angles.
Finally, sensitivity of EDR to canopy clumping is nearly identical to that of leaf area index, which makes sense given the interaction between these terms in defining canopy transmissivity (Equations~\ref{eq:tau_r} and~\ref{eq:tai}).

% \begin{figure}
%   \centering
%   \includegraphics{4_edr/figures/explore_spectra/edr_wood_compare.pdf}
%   \caption{\
%     Comparison of predicted canopy reflectance by 4SAIL (directional-hemispherical) and EDR with and without wood reflectance included.
%   }\label{fig:wood_compare}
% \end{figure}

Compared to 4SAIL, EDR consistently overpredicts canopy reflectance across the entire spectrum (Figures~\ref{fig:sensitivity_leaf_single},~\ref{fig:sensitivity_structure_single}, and~\ref{fig:wood_compare}).
A significant part of this bias can be explained by the inclusion of wood reflectance in EDR, but a persistent positive bias remains across most of the spectrum even after setting wood reflectance to zero (Figure~\ref{fig:wood_compare}).

% \begin{figure}
%   \centering
%   \includegraphics[width=\textwidth]{4_edr/figures/explore_spectra/edr_sensitivity_double.pdf}
%   \caption{\
%     Sensitivity of EDR canopy reflectance to leaf water content of top (\textit{left}) and bottom (\textit{right}) cohorts within a multi-cohort canopy,
%     when the top canopy is closed (\textit{top}) and highly clumped (\textit{bottom}).
%     Top and bottom cohorts are, respectively, Early and North Mid Hardwood with DBH 40 and 30, LAI 2.4 and 1.3, and equal stem density.
%     Clumping factors for closed and clumped canopies are 0.9 and 0.3, respectively.
%   }\label{fig:sensitivity_water_multi}
% \end{figure}

EDR canopy reflectance is highly sensitive to the properties of the tallest cohort, and shows virtually no sensitivity to the optical properties of lower cohorts (Figure~\ref{fig:sensitivity_water_multi}).
Clumping (or reduced LAI) allow more light to penetrate the canopy and therefore increases the sensitivity of canopy reflectance to the properties of lower layers, but this sensitivity is effect is still significantly muted compared to the top canopy.

\section{Discussion}

Calibrating and validating vegetation models using optical remote sensing data typically involves derived data products (e.g., MODIS GPP) that rely on their own models;
in other words, ``bringing the observations closer to the models''.
In this study, we presented an alternative approach whereby we bring the models closer to the observations by training a vegetation model to simulate full-range hyperspectral surface albedo as observed by optical remote sensing instruments.
We then demonstrated how this approach could be used to calibrate the model against airborne imaging spectroscopy data from AVIRIS-Classic.
We found that such calibration reduced uncertainties in parameters related to leaf biochemistry and canopy structure, even for parameters with well-informed priors (Figure~\ref{fig:posterior-pft}).
Moreover, we found that that the calibrated model was able to reproduce observed surface albedo
(Figures~\ref{fig:spec-error-aggregate},~\ref{fig:bias-boxplot-pft},~\ref{fig:spec-error-all}, and~\ref{fig:spec-error-allsites})
reasonably well across large number of geographically (Figure~\ref{fig:site-map}), structurally, and compositionally (Figure~\ref{fig:site-structure}) diverse sites.
However, the calibrated model underpredicted LAI at sites with mostly small trees and overpredicted LAI at sites with mostly large trees (Figures~\ref{fig:lai-pred-obs} and~\ref{fig:lai-bias-dbh-bypft}).

Compared to previous similar efforts that have coupled vegetation models to external canopy radiative transfer models~\citep{knorr2001assimilation, nouvellon2001coupling, quaife2008assimilating},
our work is novel because it uses a canopy radiative transfer formulation that \emph{already exists inside the model itself}.
This reduces the number of new assumptions and variables we have to introduce and increases the extent to which constraint on canopy radiative transfer parameters propagates to other related processes in the model.
More importantly, in such a coupling, the only way that observed reflectance constrains the model is through the foliar biomass, and additional information from the reflectance on canopy structure is confined to the GORT parameters.
By contrast, in our approach, parameters and states in the shortwave canopy radiative transfer submodel also influence other model processes, including thermal radiative transfer, micrometeorology, and competition~\citep{longo2019ed2description}, with profound consequences for model predictions of ecosystem fluxes and composition~\citep{viskari_2019_influence}.

Cross-validation and out-of-sample validation are useful tests of model performance, and we recommend these activities as future directions for this and similar work.
However, because our calibration was joint across all sites, the marginal benefit of a separate validation at other sites not used in the calibration was relatively low.
With 54 sites in our calibration, any single site represents <2\% of the data, and for a joint calibration without site random effects, we have every reason to believe that the calibration is not overfitting to any individual site.
Trying to fit any one site well would cause others to do worse (especially given the large observed variability in forest structure) unless the EDR model structure was reasonable and the parameters chosen were genuinely good choices.

In this study, the vegetation composition at each site (including the PFT distribution and size-age structure) was prescribed in detail based on inventory data.
This allowed us to focus the calibration on model parameters related canopy radiative transfer model parameters.
However, ED2 is a dynamic vegetation model whose core purpose is to predict how vegetation composition and structure evolve through time.
An important future direction of this work is to evaluate such dynamic ED2 simulations where vegetation composition and structure and predicted with some uncertainty.
In doing so, there is an opportunity to further tighten the link between canopy radiative transfer and other model processes.
For example, the PROSPECT leaf water content parameter (\emph{Cw}) provides a physical link between leaf optical properties and hydraulics and could therefore be used as a constraint in ED2's recently developed dynamic hydraulics module~\citep{xu2021leaf}.

The canopy radiative transfer model in ED2 is derived from the two-stream model of \citet{sellers1985canopy} and adapted to a multi-level canopy.
Similar versions of this two-stream formulation are present in other land surface models, including CLM~\citep{clm45_note}, SiB~\citep{baker2008seasonal}, Noah~\citep{niu2011community}, tRIBS-VEGGIE~\citep{ivanov2008vegetationhydrology}, IMOGEN~\citep{huntingford2008quantifying}, and JULES~\citep{best_2011_joint}.
Although the exact parameterization and implementation differs across these models, the similarity of the underlying conceptual framework and radiative transfer coefficients means that our approach should be directly transferable to all of these models.

Nevertheless, our analysis echoed some known challenges in canopy radiative transfer modeling.
One challenge is equifinality in the contributions of leaf biochemistry, leaf morphology, and different aspects of canopy structure to canopy albedo, which means that multiple variable and parameter combinations can produce very similar canopy albedo responses (\citealt{lewis2007spectral}; Figures~\ref{fig:edr-sensitivity-lai}--\ref{fig:edr-sensitivity-orient}).
We mitigated the equifinality between leaf traits and canopy structure by using informative priors on leaf traits from an independent data source~\citep{shiklomanov_dissertation}.
However, there is additional equifinality in the effects of the EDR canopy structure parameters.
For example, because the effective LAI used in EDR’s actual radiative transfer calculations is defined as the product of ``true'' LAI and clumping factor (equation~\ref{eqn:elai}), and because LAI is, in turn, derived from multiple parameters (leaf biomass allometry, specific leaf area; equation~\ref{eqn:lai}), these parameters collectively cannot be independently determined from reflectance data alone.
At the same time, increasing the leaf orientation factor (more horizontal, or ``planophile'', leaf orientation) has a similar (although not identical) effect to increasing LAI and clumping factor---namely, increasing canopy reflectance, especially in the near-infrared (Figure~\ref{fig:edr-sensitivity-orient}).
Collectively, these issues may help explain some of the edge-hitting behavior (parameter distributions clustered at the ends of the distribution) observed in our posterior estimates (Figure~\ref{fig:posterior-pft}), and some of the bias in our LAI estimates (Figure~\ref{fig:lai-pred-obs}).
In future work, we suggest combining our approach with additional kinds of remote sensing measurements capable of directly constraining these structural parameters such as waveform LiDAR (which can provide a robust constraint on the canopy structural profile; \citealt{ferraz2020tropical}) to reduce equifinality.

That being said, one major advantage of the Bayesian calibration approach is that its output is a joint posterior distribution that includes not only fully quantified uncertainties for each parameter but also the variance-covariance matrix of each parameter.
Equifinality in parameters would manifest as strong pairwise correlation between parameters in the posterior distribution.
Examining this correlation matrix shows that there are some parameter pairs with strong correlations, such as the hypothesized negative correlations between leaf allometry and clumping factor across most PFTs (Figure~\ref{fig:posterior-correlations}).
However, these correlations do not occur in all parameters that exhibited edge-hitting behavior.
For instance, clumping factor also exhibited edge-hitting behavior for mid-successional hardwood PFTs (Figure~\ref{fig:posterior-pft}), but the corresponding correlation coefficient was only weakly negative (Figure~\ref{fig:posterior-correlations}).
Strong correlations also occurred among some of the PROSPECT parameters, and between PROSPECT and structural parameters, but contributed little to equifinality because the strong constraints on PROSPECT led to overall small covariance terms (results not shown).
Finally, because our calibration captured all of these covariances, the presence of moderate equifinality did not preclude ecologically meaningful parameter constraints or accurate predictions because these covariances are being propagated into predictions.
This is directly analogous to how a linear regression can have a tight confidence interval, despite high correlations between the slope and intercept, with that equifinality driving the characteristic hourglass shape of a regression confidence interval.

EDR tended to underpredict LAI at high-density sites with low mean DBH and overpredict LAI at low-density sites with high mean DBH (Figures~\ref{fig:lai-pred-obs},~\ref{fig:lai-bias-dbh-bypft}, and~\ref{fig:lai-bias-dens-bypft}).
The relationship between DBH and LAI is controlled primarily by the leaf biomass allometry, which in EDR is fixed at the PFT level (equation~\ref{eqn:bleaf}).
This fixed relationship neglects the known inter- and intra-specific variability in tree allocation strategies~\citep{forrester2017generalized, dolezal2020contrasting}.
For example, \citet{forrester2017generalized} show that the relationship between DBH and foliar biomass is modulated by tree age, stand density, and climate variables, none of which are accounted for in the ED2 allometry routines.
This variability can be incorporated directly into ED2 by making allometry parameters dynamic functions of some of the aforementioned covariates,
or indirectly via hierarchical calibration whereby model parameters vary across sites~\citep{dietze2008capturing}.
Overall, this analysis reiterates the importance of evaluating models against multiple distinct variables---after all, none of these biases would have been apparent from looking at the reflectance simulations alone.

Our analysis also revealed some structural issues with EDR itself.
EDR consistently predicted lower hemispherical reflectance than SAIL (Figures~\ref{fig:edr-sail-comparison-czen} and~\ref{fig:edr-sail-comparison-lai}).
This difference can be attributed primarily to differences in how each model defines the direct radiation backscatter coefficient in the radiative transfer equation.
A detailed description of the discrepancy is provided in~\citet{yuan2017reexamination}.
Briefly, EDR (and the \citealt{sellers1985canopy} model from which EDR is derived) defines direct radiation backscatter as a function of the single-scattering albedo (equation~\ref{eqn:backscatter-direct}),
which in turn is a challenging integral involving the leaf scattering phase function and leaf projected area function (equation~\ref{eqn:ssa-integral}).
The analytical solution to this integral in EDR (equation~\ref{eqn:ssa-solved}) assumes a uniform scattering phase function, which is appropriate for point scatterers but less so for horizontal surfaces like leaves.
The practical consequence of this assumption is a lower value of the direct radiation backscatter and therefore a lower albedo, which is consistent with the results of our sensitivity analysis.
This underestimation of albedo described above may also help explain the edge-hitting behavior in our posterior distributions (Figure~\ref{fig:posterior-pft}) as well as the relatively low accuracy of our LAI estimates (Figure~\ref{fig:lai-pred-obs}).
Specifically, our EDR calibration may be trying to compensate for underestimated albedos via a tendency to prefer
higher effective LAI values (which results in higher values of the leaf biomass allometry and clumping factor for some PFTs; e.g., early and mid hardwood and northern pine in Figure~\ref{fig:posterior-pft}) and
more horizontal leaf distributions (i.e., higher leaf orientation factor; e.g., late hardwood in Figure~\ref{fig:posterior-pft}), both of which increase albedo (Figures~\ref{fig:edr-sensitivity-lai}--\ref{fig:edr-sensitivity-orient}).

Meanwhile, the SAIL definition of direct backscatter is a more simple function of leaf scattering, leaf angle distribution, and canopy optical depth that also produces a more accurate albedo estimate~\citep{yuan2017reexamination}.
Given that underestimating albedo can have significant consequences for the biological, ecological, and physical predictions of ED2~\citep{viskari_2019_influence}, incorporating this fix into the ED2 canopy RTM is an important future direction of our work.
However, doing so is outside the scope of this work because doing so would require propagating the different coefficients through the complex, multiple-canopy-layer solution of EDR (Section~\ref{subsubsec:multi-canopy-solution})---a non-trivial task.

A significant caveat to the broader application of our approach is that there is a subtle but significant difference between the physical quantity EDR predicts and the quantities typically measured by optical remote sensing.
Specifically, EDR predicts the hemispherical reflectance---the ratio of total radiation leaving the surface to the total radiation incident upon the surface, integrated over all viewing angles (a.k.a, ``blue-sky albedo'').
On the other hand, optical remote sensing platforms typically measure the directional reflectance factor---the ratio of actual radiation reflected by a surface to the radiation reflected from an ideal diffuse scattering source (e.g., a Spectralon calibration panel) subject to the same illumination, in a specific viewing direction~\citep{schaepman-strub2006reflectance}.
These two quantities are numerically equivalent only for ideal Lambertian surfaces or, for non-Lambertian surfaces, under specific sun-sensor geometries.
However, vegetation canopies---the focus of this study---are known to exhibit reflectance with very strong angular dependence, so a comparison of canopy hemispherical reflectance with a remotely sensed directional reflectance factor is invalid.

Our specific analysis is valid because---as described in the Methods (Section~\ref{subsec:site-data})---we used AVIRIS data that were also BRDF-corrected,
whereby the directional reflectance estimates from the atmospheric correction process were further converted to estimates of hemispherical reflectance via a polynomial approximation of the Ross-Li semiempirical BRDF model~\citep{lucht2000algorithm}.
Another dataset that would have been valid for our analysis (albeit, one with much lower spatial and spectral resolution) is the MODIS albedo product (MOD43), which takes advantage of the angular sampling of the MODIS instrument to quantify the surface BRDF characteristics and therefore more precisely estimate the albedo~\citep{wang2004using, schaaf2015mcd43a1}.
However, our approach as described here would \emph{not} be valid for the surface reflectance products produced by nadir-viewing instruments such as Landsat, Sentinel-2, or most airborne platforms, at least without additional processing steps on the data,
or, preferably, modification of the underlying radiative transfer models to allow for the prediction of directional reflectance.
Fortunately, the assumptions and parameters that comprise two-stream radiative transfer models like EDR and its parent model~\citep{sellers1985canopy} are readily adaptable to prediction of directional reflectance.
For example, the SAIL model~\citep{verhoef1984light, verhoef2007coupled}---which predicts both hemispherical and directional reflectance, and which has a long history of successful application to remote sensing---makes the same general assumptions as EDR and even shares many of the underlying coefficients~\citep{yuan2017reexamination}.
Alternatively, land surface models can take advantage of recent advances in radiative transfer theory to improve their accuracy without significant computational penalty~\citep[e.g.,][]{hogan_2018_fast}.

A related issue is the missing or simplistic treatment of two- and three-dimensional heterogeneity in canopy structure in EDR.\@
For one, the treatment of leaves as infinitely small elements randomly distributed through the canopy space---a common feature of all two-stream approximations---neglects complex realities of the canopy light environment such as gaps and self-shading.
In EDR, self-shading is handled via the clumping factor parameter, which functions as a scalar correction on the leaf area index (Equation~\ref{eqn:elai}).
A key feature of EDR design is its representation of multiple co-existing plant cohorts competing for light within a single patch;
however, horizontal heterogeneity and competition between these cohorts is ignored.
Improved representation of lateral energy transfer can improve the accuracy of simulations of the canopy light environment, and recent theoretical advances show that this can be accomplished without a significant loss in computational performance~\citep{hogan_2018_fast}.
Treatment of horizontal competition also plays an important role in the outcomes of competition for light between different plants~\citep{fisher2018vegetation}.
A useful avenue for development and parameterization of these models is comparison to more sophisticated and realistic three-dimensional representations of radiative transfer~\citep[e.g.][]{widlowski2007third}, which are themselves too computationally demanding to be coupled to ecosystem models, but from which empirical distributions and response functions could be derived and against which the behavior of simpler models could be evaluated.

\conclusions

Remote sensing observations are unrivaled in their spatial completeness and extent, notably extending to regions like the tropics and high latitudes that are relatively undersampled but have a disproportionate impact on the global climate system~\citep{schimel2015observing} and/or global biodiversity~\citep{jetz_2016_monitoring}.
At the same time, satellite time series provide multidecadal records with relatively high temporal frequency, which have tremendous utility for calibrating model projections of past ecological dynamics~\citep{kennedy2014bringing, pasquarella2016imagery}.
Used in combination with other emerging data sources, including global trait databases and eddy covariance measurements, remote sensing can be a transformative force in ecosystem ecology.

In this paper, we showed that using a vegetation model to directly simulate surface reflectance is a promising approach for calibrating and validating models against remotely sensed observations.
To do this, we modified the ED2 dynamic vegetation model to predict full-range hyperspectral hemispherical surface reflectance and then calibrated this modified model against airborne, BRDF-corrected imaging spectroscopy data.
The calibration successfully reduced uncertainties in model parameters related to canopy structure and leaf biogeochemistry for five plant functional types for five plant functional types characteristic of temperate forests of the northeastern United States.
The calibrated model was able to accurately reproduce observed surface reflectance for sites with highly varied forest composition and structure using a single common set of parameters (i.e., not site-specific parameters).
However, the calibrated model predicted leaf area index values that did not agree well with observations and had parameter estimates that exhibited edge-hitting behavior, both of which suggest structural issues in the model.
Comparison against a canopy radiative transfer model commonly used in the remote sensing community~\citep[PRO4SAIL,][]{verhoef2007coupled} suggested that our model may be systematically underpredicting surface albedo.
Given the direct role albedo plays in the canopy light and thermal environment in ED2, this bias could have significant downstream consequences for ED2 predictions of physiological and ecological processes.
We therefore recommend structural changes to the ED2 canopy radiative transfer model to resolve this bias, and recommend calibrating the updated model against remotely sensed surface reflectance, as we demonstrated here.
We note that the basic structure and assumptions of the ED2 canopy radiative transfer scheme are shared by many other vegetation models,
so we expect that both this issue and our recommendations for resolving it are highly transferable within the vegetation modeling community.
More generally, we recommend the development of additional ``observation operators'' similar to ours for other classes of remote sensing data, such as thermal, microwave, and LiDAR, in ED2 and other dynamic vegetation models to allow these models to take full advantage of remote sensing observations.


%% The following commands are for the statements about the availability of data sets and/or software code corresponding to the manuscript.
%% It is strongly recommended to make use of these sections in case data sets and/or software code have been part of your research the article is based on.

\codedataavailability{
  All of the code and data required to reproduce this study are publicly available via an Open Science Framework (OSF) project, located at
  \url{https://osf.io/b6umf/}.
} %% use this section when having data sets and software code available

\appendix
\section{Supplementary figures}

\begin{figure}[ht]
  \centering
  \includegraphics[width=\textwidth]{figures/posterior-soil}
  \caption{\label{fig:posterior-soil}\
    Site-specific relative soil moisture (0 = dry, 1 = wet) posterior estimates.
    Sites are sorted in order of increasing weighted evergreen fraction.
  }
\end{figure}

\clearpage

\begin{figure}
  \centering
  \includegraphics[width=\textwidth,height=0.8\textheight,keepaspectratio]{figures/spec-error-all}
  \caption{\label{fig:spec-error-allsites}\
    Comparison between AVIRIS observed (black) and
    surface reflectance for each site used in the calibration.
    Sites are sorted in order of decreasing mean difference between observed and EDR predicted reflectance
    (largest underestimates first, largest overestimates last).
  }
\end{figure}

\clearpage

\begin{figure}
  \centering
  \includegraphics[width=\textwidth]{figures/bias-density-pft}
  \caption{\label{fig:bias-density-pft}\
    Mean reflectance bias (EDR predicted $-$ observed) for each by spectral region and dominant plant functional type (PFT) as a function of site stem density.
    PFTs are abbreviated as follows:
    EH:\@Early Hardwood;
    MH:\@North Mid Hardwood;
    LH:\@Late Hardwood;
    NP:\@Northern Pine;
    LC:\@Late conifer
  }
\end{figure}

\clearpage

\begin{figure}[ht]
  \centering
  \includegraphics[width=\textwidth,height=0.8\textheight,keepaspectratio]{figures/tree-sites-q0}
  \caption{\label{fig:tree-sites-q0}\
    EDR predicted vs.\ observed spectra and species composition for the first quartile of sites by DBH.\@
  }
\end{figure}

\clearpage

\begin{figure}[ht]
  \centering
  \includegraphics[width=\textwidth,height=0.8\textheight,keepaspectratio]{figures/tree-sites-q25}
  \caption{\label{fig:tree-sites-q25}\
    As above, but for the second quartile of sites by DBH.\@
  }
\end{figure}

\clearpage

\begin{figure}[ht]
  \centering
  \includegraphics[width=\textwidth,height=0.8\textheight,keepaspectratio]{figures/tree-sites-q50}
  \caption{\label{fig:tree-sites-q50}\
    As above, but for the third quartile of sites by DBH.\@
  }
\end{figure}

\clearpage

\begin{figure}[ht]
  \centering
  \includegraphics[width=\textwidth,height=0.8\textheight,keepaspectratio]{figures/tree-sites-q75}
  \caption{\label{fig:tree-sites-q75}\
    As above, but for the fourth quartile of sites by DBH.\@
  }
\end{figure}

\clearpage

\begin{figure}[ht]
  \centering
  \includegraphics[width=\textwidth,height=0.8\textheight,keepaspectratio]{figures/tree-sites-Early_Hardwood}
  \caption{\label{fig:tree-sites-EH}\
    As above, but for sites where Early Hardwood trees had the largest mean DBH.\@
  }
\end{figure}

\clearpage

\begin{figure}[ht]
  \centering
  \includegraphics[width=\textwidth,height=0.8\textheight,keepaspectratio]{figures/tree-sites-North_Mid_Hardwood}
  \caption{\label{fig:tree-sites-MH}\
    As above, but for sites where Mid Hardwood trees had the largest mean DBH.\@
  }
\end{figure}

\clearpage

\begin{figure}[ht]
  \centering
  \includegraphics[width=\textwidth,height=0.8\textheight,keepaspectratio]{figures/tree-sites-Late_Hardwood}
  \caption{\label{fig:tree-sites-LH}\
    As above, but for sites where Late Hardwood trees had the largest mean DBH.\@
  }
\end{figure}

\clearpage

\begin{figure}[ht]
  \centering
  \includegraphics[width=\textwidth,height=0.8\textheight,keepaspectratio]{figures/tree-sites-Northern_Pine}
  \caption{\label{fig:tree-sites-P}\
    As above, but for sites where Pine trees had the largest mean DBH.\@
  }
\end{figure}

\clearpage

\begin{figure}[ht]
  \centering
  \includegraphics[width=\textwidth,height=0.8\textheight,keepaspectratio]{figures/tree-sites-Late_Conifer}
  \caption{\label{fig:tree-sites-LC}\
    As above, but for sites where Late Conifer trees had the largest mean DBH.\@
  }
\end{figure}

\clearpage

\begin{figure}[ht]
  \centering
  \includegraphics[width=\textwidth]{figures/edr-sail-comparison-lai}
  \caption{\label{fig:edr-sail-comparison-lai}\
    Same as Figure~\ref{fig:edr-sail-comparison-czen}, but varying leaf area index (LAI) and fixing $\cos(\theta_{s}) = 0.85$, a typical value for our study.
  }
\end{figure}


\noappendix       %% use this to mark the end of the appendix section. Otherwise the figures might be numbered incorrectly (e.g. 10 instead of 1).

%% Regarding figures and tables in appendices, the following two options are possible depending on your general handling of figures and tables in the manuscript environment:

%% Option 1: If you sorted all figures and tables into the sections of the text, please also sort the appendix figures and appendix tables into the respective appendix sections.
%% They will be correctly named automatically.

%% Option 2: If you put all figures after the reference list, please insert appendix tables and figures after the normal tables and figures.
%% To rename them correctly to A1, A2, etc., please add the following commands in front of them:

% \appendixfigures  %% needs to be added in front of appendix figures

% \appendixtables   %% needs to be added in front of appendix tables

%% Please add \clearpage between each table and/or figure. Further guidelines on figures and tables can be found below.

\authorcontribution{
  ANS led the analysis and manuscript preparation.
  MCD and SPS conceived of the original idea, participated in regular discussions about the study with ANS, and provided funding and infrastructure support.
  IF provided formatted input data on site structure and composition.
  TV developed the original version of the EDR code.
  All authors reviewed the manuscript draft and contributed revisions and suggestions.
} %% this section is mandatory

\competinginterests{%
  The authors declare no competing interests.
} %% this section is mandatory even if you declare that no competing interests are present

\begin{acknowledgements}
  This work was supported financially by
  NASA awards NNX14AH65G, NNX16AO13H, and NNG20OB24A,
  by NSF Awards 1655095 and 1457890,
  and by NASA's Surface Biology and Geology (SBG) mission study.
  S.P.S.\ was also supported by the United States Department of Energy contract No.\ DE-SC0012704 to Brookhaven National Laboratory.
  Cyberinfrastructure for this work was provided by the
  Boston University Department of Earth and Environment,
  Brookhaven National Laboratory,
  Pacific Northwest National Laboratory,
  and NASA High-End Computing (HEC).
  We would also like to thank Dr.\ Tristan Quaife and one anonymous reviewer for their helpful feedback on the first version of this manuscript.
\end{acknowledgements}


%% REFERENCES

%% The reference list is compiled as follows:

% \begin{thebibliography}{}

% \bibitem[AUTHOR(YEAR)]{LABEL1}
% REFERENCE 1

% \bibitem[AUTHOR(YEAR)]{LABEL2}
% REFERENCE 2

% \end{thebibliography}

%% Since the Copernicus LaTeX package includes the BibTeX style file copernicus.bst,
%% authors experienced with BibTeX only have to include the following two lines:
%%
\bibliographystyle{copernicus}
\bibliography{library.bib,lib2.bib}
%%
%% URLs and DOIs can be entered in your BibTeX file as:
%%
%% URL = {http://www.xyz.org/~jones/idx_g.htm}
%% DOI = {10.5194/xyz}


%% LITERATURE CITATIONS
%%
%% command                        & example result
%% \citet{jones90}|               & Jones et al. (1990)
%% \citep{jones90}|               & (Jones et al., 1990)
%% \citep{jones90,jones93}|       & (Jones et al., 1990, 1993)
%% \citep[p.~32]{jones90}|        & (Jones et al., 1990, p.~32)
%% \citep[e.g.,][]{jones90}|      & (e.g., Jones et al., 1990)
%% \citep[e.g.,][p.~32]{jones90}| & (e.g., Jones et al., 1990, p.~32)
%% \citeauthor{jones90}|          & Jones et al.
%% \citeyear{jones90}|            & 1990



%% FIGURES

%% When figures and tables are placed at the end of the MS (article in one-column style), please add \clearpage
%% between bibliography and first table and/or figure as well as between each table and/or figure.

% The figure files should be labelled correctly with Arabic numerals (e.g. fig01.jpg, fig02.png).


%% ONE-COLUMN FIGURES

%%f
%\begin{figure}[t]
%\includegraphics[width=8.3cm]{FILE NAME}
%\caption{TEXT}
%\end{figure}
%
%%% TWO-COLUMN FIGURES
%
%%f
%\begin{figure*}[t]
%\includegraphics[width=12cm]{FILE NAME}
%\caption{TEXT}
%\end{figure*}
%
%
%%% TABLES
%%%
%%% The different columns must be seperated with a & command and should
%%% end with \\ to identify the column brake.
%
%%% ONE-COLUMN TABLE
%
%%t
%\begin{table}[t]
%\caption{TEXT}
%\begin{tabular}{column = lcr}
%\tophline
%
%\middlehline
%
%\bottomhline
%\end{tabular}
%\belowtable{} % Table Footnotes
%\end{table}
%
%%% TWO-COLUMN TABLE
%
%%t
%\begin{table*}[t]
%\caption{TEXT}
%\begin{tabular}{column = lcr}
%\tophline
%
%\middlehline
%
%\bottomhline
%\end{tabular}
%\belowtable{} % Table Footnotes
%\end{table*}
%
%%% LANDSCAPE TABLE
%
%%t
%\begin{sidewaystable*}[t]
%\caption{TEXT}
%\begin{tabular}{column = lcr}
%\tophline
%
%\middlehline
%
%\bottomhline
%\end{tabular}
%\belowtable{} % Table Footnotes
%\end{sidewaystable*}
%
%
%%% MATHEMATICAL EXPRESSIONS
%
%%% All papers typeset by Copernicus Publications follow the math typesetting regulations
%%% given by the IUPAC Green Book (IUPAC: Quantities, Units and Symbols in Physical Chemistry,
%%% 2nd Edn., Blackwell Science, available at: http://old.iupac.org/publications/books/gbook/green_book_2ed.pdf, 1993).
%%%
%%% Physical quantities/variables are typeset in italic font (t for time, T for Temperature)
%%% Indices which are not defined are typeset in italic font (x, y, z, a, b, c)
%%% Items/objects which are defined are typeset in roman font (Car A, Car B)
%%% Descriptions/specifications which are defined by itself are typeset in roman font (abs, rel, ref, tot, net, ice)
%%% Abbreviations from 2 letters are typeset in roman font (RH, LAI)
%%% Vectors are identified in bold italic font using \vec{x}
%%% Matrices are identified in bold roman font
%%% Multiplication signs are typeset using the LaTeX commands \times (for vector products, grids, and exponential notations) or \cdot
%%% The character * should not be applied as mutliplication sign
%
%
%%% EQUATIONS
%
%%% Single-row equation
%
%\begin{equation}
%
%\end{equation}
%
%%% Multiline equation
%
%\begin{align}
%& 3 + 5 = 8\\
%& 3 + 5 = 8\\
%& 3 + 5 = 8
%\end{align}
%
%
%%% MATRICES
%
%\begin{matrix}
%x & y & z\\
%x & y & z\\
%x & y & z\\
%\end{matrix}
%
%
%%% ALGORITHM
%
%\begin{algorithm}
%\caption{...}
%\label{a1}
%\begin{algorithmic}
%...
%\end{algorithmic}
%\end{algorithm}
%
%
%%% CHEMICAL FORMULAS AND REACTIONS
%
%%% For formulas embedded in the text, please use \chem{}
%
%%% The reaction environment creates labels including the letter R, i.e. (R1), (R2), etc.
%
%\begin{reaction}
%%% \rightarrow should be used for normal (one-way) chemical reactions
%%% \rightleftharpoons should be used for equilibria
%%% \leftrightarrow should be used for resonance structures
%\end{reaction}
%
%
%%% PHYSICAL UNITS
%%%
%%% Please use \unit{} and apply the exponential notation


\end{document}
