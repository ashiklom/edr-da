\introduction  %% \introduction[modified heading if necessary]

Dynamic vegetation models play a vital role in modern terrestrial ecology, and Earth science more generally.
The terrestrial carbon cycle is a major biogeochemical feedback in the global climate system~\citep{heinze_2019_esd_reviews}, and accurate predictions of terrestrial carbon cycling rely on accurate representations of vegetation dynamics~\citep{pacala_1995_details_that_matter}.
Vegetation also plays an important role in the water cycle and surface energy balance, with major climate implications~\citep{bonan2008forests}.
In addition, the distribution of tree species, the structure of plant canopies, and many other variables simulated by dynamic vegetation models are also important predictors of biodiversity, making vegetation models an important tool for conservation~\citep{mcmahon2011improving}.
Effective calibration and validation of model projections is therefore of broad concern.

Many previous efforts have used various data streams to calibrate or constrain dynamic vegetation model parameters and states.
Among these data streams, remote sensing is particularly promising due to its consistent measurement methodology and largely uninterrupted global coverage.
Data products derived from remote sensing observations have been used to inform, among others,
phenology~\citep{knorr2010carbon, viskari2015modeldata}
and absorbed photosynthetically-active radiation~\citep{peylin2016new, schurmann2016constraining, zobitz_2014_joint}.
% and primary productivity~\citep{}. %TODO Different reference here?
However, there are issues with using derived remote sensing products to calibrate vegetation models.
The relationships between remotely sensed surface reflectance and vegetation structure and function are complex and multifaceted.
Simple polynomial relationships between spectral indices (e.g., Normalized Difference Vegetation Index, NDVI; Enhanced Vegetation Index, EVI) and vegetation properties (e.g., leaf area index, LAI) are often confounded by other ecosystem characteristics, including soil~\citep{myneni1994relationship} and snow~\citep{zhang2020evaluating}, or sensor configuration~\citep{fensholt2004evaluation}.
More sophisticated approaches for estimating vegetation properties based on physically-based radiative transfer models face issues of equifinality, whereby many different combinations of vegetation and soil properties can ultimately produce the same modeled surface reflectance~\citep{combal2003retrieval, lewis2007spectral}.
Meanwhile, estimating quantities with more indirect relationships to surface reflectance, such as rates of primary productivity, requires a number of assumptions about resource use efficiency and other factors~\citep{running2004continuous} that can introduce considerable uncertainty and bias into the estimates.
Collectively, these issues help explain the large differences in estimates of surface characteristics across different remote sensing instruments~\citep{liu_2018_satellite}.
Robust, pixel-level uncertainty estimates for remote sensing data products would help alleviate some of these concerns, but such estimates are not widely available for most data products.

One way to overcome these limitations of derived remote sensing data products while still leveraging the capabilities of remote sensing is to work with lower-level surface reflectance products.
This can be accomplished by coupling dynamic vegetation models with leaf and canopy radiative transfer models that simulate surface reflectance as a function of known surface characteristics~\citep{knorr2001assimilation, nouvellon2001coupling, quaife2008assimilating}.
Such an approach draws on decades of research on simulation of vegetation optical properties given their structural and biochemical characteristics~\citep{dickinson_1983_land, sellers1985canopy, verhoef1984light, lewis2007spectral, jacquemoud2009prospect, pinty2004synergy, widlowski2007third, widlowski2015fourth, hogan_2018_fast} while avoiding the computational and conceptual challenges of inverse parameter estimation in radiative transfer modeling~\citep{combal2003retrieval, lewis2007spectral}.
Moreover, the ability to simulate dynamics of surface reflectance in response to changes in ecosystem properties is valuable even independently of its utility for remote sensing data assimilation, as vegetation-induced changes in surface reflectance exert a strong influence on climate~\citep{bonan2008forests, swann2010changes, swann2012midlatitude}.

However, many land surface models already include their own internal canopy radiative transfer calculations~\citep{dickinson1983land, sellers1985canopy}, which can be used for a more direct comparison to remotely sensed surface reflectance.
These calculations are necessary to simulate impact of vegetation on surface energy balance~\citep{bonan2008forests} and to accurately model plant function, which is fundamentally driven by light~\citep{hikosaka1995model, robakowski_2004_growth, niinemets2016withincanopy, keenan2016global}.
Canopy radiative transfer plays a particularly important role in the current generation of demographically-enabled dynamic vegetation models, where differences in canopy radiative transfer representations and parametrizations have major impacts on predicted community composition and biogeochemistry~\citep{loew_2014_do, fisher2018vegetation, viskari_2019_influence}.
To date, assimilation of remotely sensed surface reflectance calculated from a vegetation model's own representation of radiative transfer has not been attempted for more complex demographically-enabled vegetation models.

Our previous work demonstrated that predictions of carbon cycling and community composition by the Ecosystem Demography model, version 2~\citep[ED2;][]{medvigy2009mechanistic} are highly sensitive to changes in parameters related to canopy structure and radiative transfer~\citep{viskari_2019_influence}.
In this study, we build on this work by calibrating and validating the ED2 model using remotely sensed surface reflectance.
First, we couple the internal ED2 canopy radiative transfer model to the PROSPECT 5 leaf radiative transfer model~\citep{feret2008prospect4} and the Hapke soil reflectance model~\citep{verhoef2007coupled} to allow ED2 to predict surface reflectance spectra at 1 \unit{nm} resolution across the complete visible-shortwave infrared (VSWIR) spectral region (400 to 2500 nm).
Second, we jointly calibrate this model at 54 sites in the US Midwest and Northeast where coincident vegetation survey data and NASA Airborne Visible/Infrared Imaging Spectrometer-Classic (AVIRIS-Classic) surface reflectance observations are available.
We hypothesize that, with known stand composition and informative priors on foliar biochemistry, calibration against airborne imaging spectroscopy will significantly constrain model parameters related to canopy structure.
Although the scope of our study is limited to the ED2 model, both the underlying size-and-age structure approximation of ED2 as well as many aspects of its canopy radiative transfer (e.g., two-stream approximation, treatment of leaf angles) are common to other land surface models~\citep[e.g., FATES;][]{koven2020benchmarking}, meaning the insights from this work more broadly applicable in model vegetation modeling.
