\section{Introduction}

Dynamic vegetation models play a vital role in modern terrestrial ecology, and Earth science more generally.
The terrestrial carbon cycle is a major biogeochemical feedback in the global climate system~\parencite{heinze_2019_esd_reviews}, and accurate predictions of terrestrial carbon cycling rely on accurate representations of vegetation dynamics (REF). % TODO Ref
Vegetation also plays an important role in the water cycle and surface energy balance, with major climate implications.
In addition, the distribution of tree species, the structure of plant canopies, and many other variables simulated by dynamic vegetation models are also important predictors of biodiversity (REF), making vegetation models an important tool for conservation %TODO Ref
Effective calibration and validation of model projections is therefore of broad concern.

% My previous chapters have shown that models have much to gain, both in terms of direct parameter constraint from trait observations and from new process representations that emerge from trait ecology more broadly.
% However, there are limits on the extent to which traits alone can improve models.
% For one, even after examining a broad range of inter- and intraspecific factors, large fractions of variability in plant function remain unexplained.
% Moreover, vegetation models are simplified abstractions of reality, with many processes omitted or represented by simplistic empirical equations with little-to-no physical basis and therefore no directly measurable trait that can serve as a parameter constraint.
% In these cases, models can only be calibrated via their emergent predictions of state variables.

Many previous efforts have used various data streams calibrate or constrain dynamic vegetation model parameters and states.
Among these data streams, remote sensing is particularly promising due to its consistent measurement methodology and largely uninterrupted global coverage.
Data products derived from remote sensing observations have been effectively used to constrain, among others,
phenology~\parencite{knorr2010carbon, viskari2015modeldata},
absorbed photosynthetically-active radiation~\parencite{peylin2016new, schurmann2016constraining},
and primary productivity~\parencite{macbean2018strong}.

However, there are issues with using derived remote sensing products to calibrate vegetation models.
The relationships between remotely sensed surface reflectance and vegetation structure and function are complex and multifaceted.
Simple polynomial relationships between spectral indices (e.g., Normalized Difference Vegetation Index, NDVI; Enhanced Vegetation Index, EVI) and vegetation properties (e.g. leaf area index, LAI) are often confounded by other ecosystem characteristics, including soil~\parencite{myneni1994relationship} and snow~\parencite{zhang2020evaluating}, or sensor configuration~\parencite{fensholt2004evaluation}.
More sophisticated approaches for estimating vegetation properties based on physically-based radiative transfer models face issues of equifinality, whereby many different combinations of vegetation and soil properties can ultimately produce the same modeled surface reflectance~\parencite{combal2003retrieval, lewis2007spectral}.
Meanwhile, the estimation of quantities with more indirect relationships to surface reflectance, such as rates of primary productivity, requires a number of assumptions about resource use efficiency and other factors~\parencite{running2004continuous} that can introduce considerable uncertainty and bias into the estimates.
Collectively, these issues help explain the large differences in estimates of surface characteristics across different remote sensing instruments~\parencite{liu_2018_satellite}.
Robust, pixel-level uncertainty estimates for remote sensing data products would help alleviate some of these concerns, but such estimates are not widely available for most data products.

One way to overcome these limitations of derived remote sensing data products while still leveraging the capabilities of remote sensing is to work directly with the observed surface reflectance.
In the context of dynamic vegetation modeling, this can be accomplished by coupling these models with leaf and canopy radiative transfer models that simulate surface reflectance as a function of known surface characteristics~\parencite{knorr2001assimilation, nouvellon2001coupling, quaife2008assimilating}.
This approach draws on decades of research on simulation of vegetation optical properties given their structural and biochemical characteristics~\parencite{dickinson_1983_land, sellers1985canopy, verhoef1984light, lewis2007spectral, jacquemoud2009prospect, pinty2004synergy, widlowski2007third, widlowski2015fourth, hogan_2018_fast}, while avoiding the computational and conceptual challenges of inverse parameter estimation~\parencite{combal2003retrieval, lewis2007spectral}.
Moreover, the ability to simulate dynamics of surface reflectance in response to changes in ecosystem properties is valuable even independently of its utility for remote sensing data assimilation, as vegetation-induced changes in surface reflectance exert a strong influence on global climate~\parencite{bonan2008forests, swann2010changes, swann2012midlatitude}.

While multiple studies have externally coupled vegetation model output to a separate canopy radiative transfer model to simulate surface reflectance~\parencite{knorr2001assimilation, nouvellon2001coupling, quaife2008assimilating},
doing this using a vegetation model's own internal representation of surface reflectance is far less common.
\Textcite{zobitz_2014_joint} used a version of the Simplified PNET model (SiPNET) with a modified representation of the fraction of photosynthetically active radiation (fAPAR) to assimilate observed fAPAR from the MODIS.
However, to the authors' knowledge, this has never been attempted for any model in the current generation of demographically-enabled dynamic vegetation models.

Notably, this is despite the fact that these models typically do include a moderately sophisticated representation of albedo.

work on vegetation model simulation and assimilation of remotely-sensed surface reflectance has involved coupling

Moreover, besides enabling assimilation of remotely sensed data, training models to accurately simulate surface reflectance is essential to properly quantifying and testing hypotheses related to vegetation-climate interactions and feedbacks.
% TODO: Dietze -- (1) This should start a new paragraph. (2) Mention that the other studies (e.g. Quaife 2008) used an external RTM, not the model's own. This is novel in this work. (3) Mention Toni's RTM uncertainty analysis in this paragraph?
For instance, the net climate effect of ongoing changes in Arctic vegetation composition depends on the balance of opposing radiative (lower albedo) and latent (increased transpiration) energy feedbacks~\parencite{swann2010changes}, so forecasting this effect requires accurate models of canopy energy transfer.
More fundamentally, light availability is a key control of photosynthesis and therefore has
immediate, direct consequences for individual plant function~\parencite{hikosaka1995model, robakowski_2004_growth, niinemets2016withincanopy, keenan2016global}
as well as longer-term, indirect consequences for competition and ecological succession~\parencite{niinemets2006tolerance, kitajima2013leaf, falster2017multitrait}.

Recognition of the importance of these processes has led to the development of vegetation models with explicit representations of canopy radiative transfer.
The most accurate canopy radiative transfer models capture both vertical and horizontal heterogeneity with very high spatial resolution~\parencite{widlowski2007third}.
However, such models are usually too computationally intensive for dynamic vegetation models, which employ various approximations based on simplifying assumptions to make the problem more tractable~\parencite{fisher2017vegetation}.
One common approach is the ``two-stream approximation'', which simplifies the problem of directional scattering within a medium by modeling the hemispherical integral of fluxes rather than individual, directional components.
In the context of radiative transfer in plant canopies, many different two-stream formulations have been developed, of which I highlight two:
One formulation was developed by Kubelka and Munk (1931)\nocite{kubelka1931article} and later adapted to vegetation canopies by Allen, Gayle, and Richardson (1970)\nocite{allen1970plantcanopy} and further refined by Suits (1971)\nocite{suits1971calculation}, Verhoef (1985)\nocite{verhoef1984light}, and others.
This theory forms the foundation of the SAIL canopy radiative transfer model~\parencite{verhoef1984light} and its derivatives~\parencite[e.g. 4SAIL][]{verhoef2007unified}, which have been used extensively in the remote sensing community for modeling and retrieving vegetation characteristics from spectral data~\cite{jacquemoud2009prospect}.
Another was developed by Meador and Weaver (1980)\nocite{meador1980twostream} for atmospheric radiative transfer, and was subsequently adapted to canopy radiative transfer by Dickinson (1983)\nocite{dickinson_1983_land} and refined by Sellers (1985)\nocite{sellers1985canopy}.
%Key assumptions of this approach are that all diffuse radiative fluxes are isotropic (i.e.\ scattering is equal in all directions) and that all canopy elements are sufficiently far from each other that there is no inter-particle shading~\parencite{meador_1980_twostream,dickinson_1983_land,SELLERS_1985_canopy}.
Due to its theoretical simplicity and low computational demand, this is the approach commonly used to represent radiative transfer in ecosystem models, including the Community Land Model~\parencite[CLM,][]{clm45_note} and the Ecosystem Demography model~\cite[ED,][]{moorcroft_2001_method, medvigy2009mechanistic, longo_2019_ed1}.
The version of this scheme used in ED2 (and derivative models) is fairly unique in its explicit representation of multiple canopy layers, which allows ED2 to simulate competition for light, a key component of modeling vegetation demographics~\parencite{fisher2017vegetation}.
However, compared to physiological processes, the structure and parameterization of canopy radiative transfer schemes in demographic models has received relatively little attention.
When canopy radiative transfer has been considered, it was shown to be important to a wide range of physiological and demographic processes.
For example, using a modified version of the ED model, Fisher et al. (2010)\nocite{fisher_2010_assessing} showed that excessive light absorption by the top cohort resulted in unrealistically excessive growth of canopy trees at the expense of understory trees.
Similarly, an analysis by Viskari et al.\ (in revision) \nocite{viskari_2019_influence} demonstrated that the Ecosystem Demography (ED2) model's predictions of ecosystem energy budget, productivity, and composition are highly sensitive to the parameterization of the model's representation of canopy radiative transfer.
Understanding and improving representations of canopy radiative transfer in dynamic vegetation models is therefore critical to accurate projections of the fate of the terrestrial biosphere.

Building on the work of Viskari et al., the objective of this chapter is to develop and demonstrate the calibration and validation of the ED2 model using remotely sensed surface reflectance.
First, I link the canopy radiative transfer model in ED2 with the PROSPECT leaf radiative transfer model to allow ED to predict full-range, hyperspectral surface reflectance at each time step.
Second, I calibrate this coupled leaf-canopy radiative transfer model at a number of sites in the US Midwest and Northeast where coincident plot vegetation survey data and observations of the NASA Airborne Visible/InfraRed Imaging Spectrometer (AVIRIS) are available.
