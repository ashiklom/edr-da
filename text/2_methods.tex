\section{Methods}

\subsection{ED2 model description}

The Ecosystem Demography version 2.2 (ED2) model simulates plot-level vegetation dynamics and biogeochemistry~\parencite{moorcroft_2001_method, medvigy2009mechanistic, longo_2019_ed1}.
By grouping individuals of similar size, structure, and composition together into cohorts, ED2 is capable of modeling patch-level competition in a computationally efficient manner.
Relevant to this work, ED2 includes a multi-layer canopy radiative transfer model that is a generalization of the two-layer two-stream radiative transfer scheme in CLM 4.5~\parencite{clm45_note}, which in turn is derived from \textcite{sellers1985canopy}.
A complete description of the model derivation is provided in the supplementary information of \textcite{longo_2019_ed1}, but for completeness, we provide an abbreviated description below:

Let $n$ be the number of cohorts in a patch.
The full canopy radiation profile in ED2 is then defined by a vector $X$ that contains two fluxes---upward ($F_{up,i}$) and downward ($F_{down,i}$)---for each cohort level $i$, plus a downward flux from the atmosphere ($F_{down,sky}$) and an upward flux from the ground ($F_{up,ground}$) surface (total size $2n + 2$):

\begin{equation}
  X =
  \begin{bmatrix}
    F_{up,ground} \\
    F_{down,1} \\
    F_{up, 1} \\
    ... \\
    F_{down,i} \\
    F_{up,i} \\
    ... \\
    F_{down,n} \\
    F_{up,n} \\
    F_{down,sky}
  \end{bmatrix}
\end{equation}

ED2 solves this vector using the following matrix equation:

\begin{equation}
  M \times X = Y
\end{equation}

where $M$ is a $(2n + 2) \times (2n + 2)$ coefficient matrix and $Y$ is a $2n + 2$ vector.
The full form of $Y$ is as follows:

\begin{equation}
  Y =
  \begin{bmatrix}
    S_0 a_{ground} () \\
    S_1 r(\psi)_1 \left[ 1 - r_0 r_1 (1 - \alpha_0) (1 - \tau_0) (1 - \alpha_1) (1 - \tau_1) \right] (1 - \tau(\psi)_1) (1 - \alpha_1) \\
    S_1 \left[ 1 - r_1 r_2 (1 - \alpha_1) (1 - \tau_1) (1 - \alpha_2) (1 - \tau_2) \right] (1 - \tau(\psi)_1) (1 - \alpha_1) (1 - r(\psi)_1) \\
    \ldots \\
    S_i r(\psi)_i \left[ i - r_{i-1} r_i (1 - \alpha_{i-1}) (1 - \tau_{i-1}) (1 - \alpha_i) (1 - \tau_i) \right] (1 - \tau(\psi)_i) (1 - \alpha_i) \\
    S_i \left[ 1 - r_i r_{i+1} (1 - \alpha_i) (i - \tau_i) (1 - \alpha_{i+1}) (1 - \tau_{i+1}) \right] (1 - \tau(\psi)_i) (1 - \alpha_i) (1 - r(\psi)_i) \\
    \ldots \\
    S_n r(\psi)_n \left[ 1 - r_{n-1} r_n (1 - \alpha_{n-1}) (1 - \tau_{n-1}) (1 - \alpha_n) (1 - \tau_n) \right] (1 - \tau(\psi)_n) (1 - \alpha_n) \\
    S_1 \left[ 1 - r_n r_{n+1} (1 - \alpha_n) (1 - \tau_n) (1 - \alpha_{n+1}) (1 - \tau_{n+1}) \right] (1 - \tau(\psi)_n) (1 - \alpha_n) (1 - r(\psi)_n) \\
    SW_{sky}
  \end{bmatrix}
\end{equation}

Here, $a_{ground}$ is the albedo of the ground under the canopy and $SW_{sky}$ is the incident shortwave hemispherical flux from the sky;
both are exogenous inputs to the model.
$S_i$ is the direct ("beam") radiation at layer $i$, and is calculated in a loop as follows:

\begin{equation}
  S_i = S_{i + 1} \tau(\psi)_i
\end{equation}

with $S_{n+1}$ as the incident direct solar flux, an exogenous input.

Other coefficients are
backscatter of direct ($r(\psi)_{i}$, given zenith angle $\psi$) and diffuse ($r_{i}$) radiation,
interception of direct ($tau(\psi)_{i}$) and diffuse ($tau_{i}$) radiation,
and absorption ($\alpha_{i}$).
Derivations of each of these coefficients is given later in this section.

The coefficient matrix $M$ is a sparse matrix with zero elements every except the diagonal and first-order off-diagonal elements; for example, for $n=3$:

\begin{equation}
  M = \begin{bmatrix}
    1 & 0 & 0 & 0 & 0 & 0 \\
    m_{2,1} & m_{2,2} & m_{2,3} & 0 & 0 & 0 \\
    0 & m_{3,2} & m_{3,3} & m_{4,3} & 0 & 0 \\
    0 & 0 & m_{4,3} & m_{4,4} & m_{4,5} & 0 \\
    0 & 0 & 0 & m_{5,4} & m_{5,5} & m_{5,6} \\
    0 & 0 & 0 & 0 & 0 & 1 \\
  \end{bmatrix}
\end{equation}

For $i = 1,2,3...n$ where $n$ is the number of cohorts, the $m$ terms are defined as follows:

\begin{align}
  \begin{split}
    m_{1,1} &= 1\\
    m_{2i,2i-1} &= - \left[ \tau_i + (1 - \tau_i)(1 - \alpha_i)(1 - r_i) \right]\\
    m_{2i,2i} &= -r_{i-1} \left[ \tau_i + (1 - \tau_i)(1 - \alpha_i)(1 - r_i) \right] (1 - \alpha_{i-1})(1 - \tau_{i-1})\\
    m_{2i,2i+1} &= 1 - r_{i-1} r_i (1 - \alpha_{i-1})(1 - \tau_{i-1})(1 - \alpha_i)(1 - \tau_i)\\
    m_{2i+1,2i} &= 1 - r_i r_{i+1} (1 - \alpha_i)(1 - \tau_i)(1 - \alpha_{i+1})(1 - \tau_{i+1})\\
    m_{2i,2i} &= -r_{i+1} \left[ \tau_i + (1 - \tau_i)(1 - \alpha_i)(1 - r_i) \right] (1 - \alpha_{i+1})(1 - \tau_{i+1})\\
    m_{2i+1,2i+2} &= - \left[ \tau_i + (1 - \tau_i)(1 - \alpha_i)(1 - r_i) \right]\\
    m_{2n+2,2n+2} &= 1
  \end{split}
\end{align}

Canopy optical property coefficients are derived as follows:

Following \textcite{clm45_note}, forward- ($\nu$) and backscattering ($\omega$) of canopy elements (leaves or stems) are defined as a function of those elements' reflectance ($R$) and transmittance ($T$; wood transmittance is assumed to be zero).
(We use index $p$ to refer to plant functional type and $p(i)$ to refer to the plant functional type of cohort $i$).

\begin{align}
  \begin{split}
    \nu_{i, leaf} &= R_{p(i), leaf} + T_{p(i), leaf}\\
    \nu_{i, wood} &= R_{p(i), wood}\\
  \end{split}
\end{align}

\begin{align}
  \begin{split}
    \omega_{i, leaf} &= \frac{R_{p(i), leaf} + T_{p(i), leaf} + \frac{1}{4} (R_{p(i), leaf}-T_{p(i), leaf})(1 - \chi)^2}{2 (R_{p(i), leaf}+T_{p(i), leaf})}\\
    \omega_{i, wood} &= \frac{R_{p(i), wood} + \frac{1}{4} (R_{p(i), wood})(1 - \chi)^2}{2 R_{p(i), wood}}
  \end{split}
\end{align}

Both of these quantities are calculated independently for leaves and wood, and then averaged based on the relative effective area of leaves ($L_{i}$) and wood ($W_{i}$) within a canopy layer.

\begin{equation}
  \nu_{i} = \nu_{i, leaf} \frac{L_{i}}{L_{i} + W_{i}} + \nu_{i, wood} (1 - \frac{L_{i}}{L_{i} + W_{i}})
\end{equation}

\begin{equation}
  \omega_{i} = \omega_{i, leaf} \frac{L_{i}}{L_{i} + W_{i}} + \omega_{wood} (1 - \frac{L_{i}}{L_{i} + W_{i}})
\end{equation}

To account for non-uniform distribution of leaves within a canopy, ED2 has a PFT-specific \emph{clumping factor} ($q$) parameter that serves as a scaling factor on leaf area index.
Therefore the effective leaf area ($L$) is related to the true leaf area index ($LAI$) by:

\begin{equation}
  L_{i} = LAI_{i} \times q_{p(i)}
\end{equation}

The directional extinction coefficient ($K(\psi)_p$)---closely related to the inverse optical depth for direct radiation ($\mu_{0,p}$)---can be expressed as:

\begin{equation}
  K(\psi)_p = \mu_{0,p}^{-1} = \frac{G(\psi)_{p}}{cos(\psi)}
\end{equation}

where $G(\psi)_{p}$ describes the mean projection per unit leaf area (or ``relative projected leaf area'') in direction $\psi$.

The leaf angle distribution function used in ED2 is the same as the one used in CLM 4.5 \parencite{clm45_note}:

\begin{equation}
  G(\psi)_{p} = \phi_{1,p} + \phi_{2,p} cos(\psi)
\end{equation}

\begin{equation}
  \phi_{1,p} = 0.5 - 0.633 \chi_{p} - 0.33 \chi_{p}^2
\end{equation}

\begin{equation}
  \phi_{2,p} = 0.877 (1 - 2 \phi_{1,p})
\end{equation}

where $\chi$ is the \emph{leaf orientation factor} parameter, defined such that -1 is perfectly vertical leaves, 1 is perfectly horizontal leaves, and 0 is randomly distributed leaf angles.

Coefficients $\phi_{1,p}$ and $\phi_{2,p}$ are also used to define the inverse optical depth for diffuse radiation per unit plant area ($\bar\mu_{p}$) (subscript $p$ is omitted from the next three equations for convenience):

\begin{equation}
  \bar\mu = \frac{1}{\phi_{2}}\left( 1 - \frac{\phi_{1}}{\phi_{2}} \ln\left( 1 + \frac{\phi_{2}}{\phi_{1}}\right) \right)
\end{equation}

The beam backscatter (or "upscatter") coefficient for direct radiation, $\beta_0$, is defined as:

\begin{equation}
  \beta_{0} = a_s(\psi) \frac{1 + \bar\mu K(\psi)}{\bar\mu_{p} K(\psi)_{p}}
\end{equation}

where $a_s(\psi)$ is the single scattering albedo coefficient, defined as (subscript $p$ dropped for simplicity):

\begin{equation}
  a_s(\psi) = \frac{1}{2}
  \frac{G(\psi)}{\cos\psi \phi_2 + G(\psi)}
  \left(
    1 -
    \frac{\cos\psi \phi_1}{\cos\psi + G(\psi)}
    \ln\left(
      \frac{\cos\psi \phi_1 + \cos\psi \phi_2 + G(\psi)}{\cos\psi \phi_1}
    \right)
  \right)
\end{equation}

(For simplicity, $a_s$ here is equivalent to $\frac{a_s}{\omega}$ in \textcite{clm45_note} equation 3.15, where $\omega$ is the leaf backscatter.)

The transmissivity of a layer to direct radiation for solar zenith angle $\psi$ ($\tau(\psi)_i$) is given by

\begin{equation}
  \tau(\psi)_i = \exp(-K(\psi)_{p(i)} TAI_i)
\end{equation}

where $TAI_{i}$ is the total plant area index (sum of effective leaf area index, $L_{i}$, and wood area index, $W_{i}$).

\subsection{ED2-PROSPECT coupling}

By default, ED2 performs canopy shortwave radiative transfer calculations for two broad spectral regions: visible (400--700 nm) and near-infrared (700--2500 nm).
For each of these regions, ED2 has user-defined prescribed leaf and wood reflectance and transmittance for each PFT, and calculates soil reflectance as the average of constant wet and dry soil reflectance values weighted by the relative soil moisture (0 = fully dry, 1 = fully wet).
In this study, we modified ED2 to perform the same canopy radiative transfer calculations but in 1 nm increments across the range 400--2500 nm.
We then simulated leaf reflectance and transmittance using the PROSPECT 5 leaf RTM,
which predicts leaf optical properties as a function of number of five parameters:
Effective number of leaf mesophyll layers (unitless, >= 1),
total chlorophyll content ($\mu$g cm$^{-2}$),
total carotenoid content ($\mu$g cm$^{-2}$),
water content (g cm$^{-2}$),
and dry matter content (g cm$^{-2}$)
\parencite{feret2008prospect4}.
For wood reflectance, we used a single representative spectrum---the mean of all wood spectra from Asner (1998), resampled to 1 nm resolution---for all PFTs. \nocite{asner_1998_biophysical}
For soil reflectance, we used the same approach as the ED2 default (weighted average of wet and dry soil reflectance), but with full 1 nm soil spectra from Hapke (REF). % TODO: Flesh this out
The final coupled PROSPECT-ED canopy radiative transfer model (hereafter known as ``EDR'') has 9 parameters for each PFT---
5 parameters for PROSPECT, specific leaf area, the intercept of the leaf allometry (slope was fixed), and clumping and orientation factors---and one site-specific parameter---the relative soil moisture.

\subsection{Sensitivity analysis}

To provide a basis for understanding the behavior of EDR, I performed a one-at-a-time sensitivity analysis to explore how its reflectance predictions vary with each leaf optical and canopy structural parameter.
To assess the mathematical foundation of the EDR canopy model (i.e.\ the Sellers two-stream scheme) without the confounding influence of multiple cohorts, I first performed this sensitivity analysis on a simulated plot containing only a single mature tree cohort.
For comparison, I also included simulations using the 4SAIL canopy radiative transfer model.
The 4SAIL model simulates four reflectance ``streams''---diffuse (hemispherical) and direct (directional) reflectance for both diffuse and direct incident radiation---\
but because EDR only simulates diffuse reflectance and its input is dominated by direct radiation (at least on sunny days, which are necessary for satellite and high-altitude airborne data), I used the ``directional-hemispherical'' output for all comparisons.
For its representation of leaf angle distribution, 4SAIL uses an ellipsoidal model that takes as input the mean leaf inclination angle ($\theta$), which is related to EDR's leaf orientation factor ($f$) by:

\begin{equation}\label{eq:orient_lidf}
  \cos \theta = \frac{1 + f}{2}
\end{equation}

Unlike EDR, 4SAIL does not account for canopy clumping.
(4SAIL does have a ``hot spot'' parameter to account for strong bi-directional reflectance effects from structurally heterogeneous canopies, but this parameter only affects the bi-directional reflectance, which was not used in this analysis).

To investigate the way EDR models interactions between canopy layers, I also performed a similar sensitivity analysis on a simulated plot with two cohorts---a dominant early successional cohort and a sub-dominant mid-successional cohort.
I examined the sensitivity of total canopy reflectance to the optical properties of both the dominant and sub-dominant cohort, and looked at how this sensitivity was affected by clumping in the upper cohort.

\subsection{Model calibration}

For model calibration, I selected 47 sites from the NASA Forest Functional Types (FFT) field campaign that contained plot-level inventory data (stem density, species identity, and DBH) coincident with observations of the NASA Airborne Visible/Infrared Imaging Spectrometer (AVIRIS).
These sites are mostly located in the United States Upper Midwest with several sites also in upstate New York and western Maryland, and include stands dominated by either evergreen or deciduous trees and spanning a wide range of structures, from dense groups of saplings (bottom right) to sparse groups of large trees (top left) (Figure~\ref{fig:sites}).
Based on ED's PFT definitions, these sites contained a total of five different temperate plant functional types: Early successional hardwood, northern mid-successional hardwood, late successional hardwood, northern pine, and late successional conifer.

% \begin{figure}
%   \centering
%   \includegraphics[width=\textwidth]{figures/sites_both.pdf}
%   \caption{\
%     Sites selected for analysis, in ``stand structure'' (\textit{main figure}) and geographic (\textit{inset}) space.
%     Colors indicate the fraction of the stand that is made up of evergreen PFTs.
%     %Large points with labels indicate sites that were selected for forward simulations.
%     % TODO: Dietze -- (1) Axes are reversed relative to Andrews ESM paper. (2) Stand density values are way too low. Basically, check against Andrews ESM paper and try to match units.
%   }\label{fig:sites}
% \end{figure}
% TODO: Change colors to be top cohort

I calibrated EDR using the same general Bayesian inversion as in Chapter 3.
The inversion fit all sites simultaneously, such that at every MCMC iteration, the algorithm proposed a set of all parameter values for each PFT and simulated spectra for each site based on its observed composition and structure.
Because of unrealistic values in the shortwave infrared spectral region in the AVIRIS observations, likely caused by faulty atmospheric correction, I only calibrated the model with observations from 400 to 1300 nm.
In addition, I changed the fixed variance model used in Chapters 2 and 3 to a two-parameter heteroskedastic variance model ($\sigma = a + bX$) to account for the fact that both model and observation errors are typically proportional to reflectance values.
To generate the initial history state files required by EDR, I ran ED2 itself for one day in midsummer (July 1), starting from vegetation initial conditions based on observed composition and structure.

For priors on the five PROSPECT parameters and specific leaf area, I performed a hierarchical multivariate analysis (see Chapter 1) on PROSPECT parameters estimated from chapter 3 and, where available, direct measurements of specific leaf area. 
For priors on the leaf biomass allometry parameters, I fit a multivariate normal distribution to allometry coefficients from Jenkins et al.~(2003, 2004) using the \texttt{PEcAn.allometry} package. \nocite{jenkins_2003_allom,jenkins_2004_allom} 
For the clumping factor, I used a uniform prior across its full range (0 to 1), and for the leaf orientation factor, I used a weakly informative re-scaled beta distribution centered on 0.5.

To alleviate issues with strong collinearity between the two allometry coefficients and the specific leaf area, I decided to remove the allometry exponent coefficient (but not the intercept) from the calibration by fixing it at its prior mean for each plant functional type.
Doing so dramatically improved the stability of the inversion algorithm and the accuracy of the results.

I evaluated the performance of the calibrated model by comparing the posterior credible intervals of modeled spectra against the AVIRIS observations at each site.
To assess the role of model structure in predictive error, I also included predictions using the 4SAIL model parameterized with the posterior means from the EDR calibration (except for clumping factor, which is absent from 4SAIL).
In addition, I compared model predictions of leaf area index (which depend on parameters calibrated in the model) against field observations.

%%% Local Variables:
%%% mode: latex
%%% TeX-master: "../dissertation"
%%% End:
