\section{Methods}

\subsection{ED2 model description}

The Ecosystem Demography version 2.2 (ED2) model simulates plot-level vegetation dynamics and biogeochemistry~\citep{moorcroft_2001_method, medvigy2009mechanistic, longo2019ed2description}.
ED2 has a fundamentally hierarchical structure:
The fundamental unit of analysis is a plant \emph{cohort}---a group of individual plants of similar size, age, and species composition (grouped into plant functional types, PFTs).
A group of cohorts with a common disturbance history (i.e., time since last disturbance) constitutes a \emph{patch},
and a group of co-located patches experiencing the same meteorological conditions constitute a \emph{site}.
At the spatial scale of this work (60 x 60m plots; see Section \ref{subsec:site-data}), we assume one patch per site.

Relevant to this work, ED2 includes a multi-layer canopy radiative transfer model that is a generalization of the two-layer two-stream radiative transfer scheme in CLM 4.5~\citep{clm45_note}, which in turn is derived from \citet{sellers1985canopy}.
At every time step, this model simulates the bi-hemispherical reflectance (BHR;\@a.k.a., ``intrinsic surface albedo'' or ``blue-sky albedo''; see \citealt{schaepman-strub2006reflectance}) as a function of that time step's vegetation composition and canopy structure.
A complete description of the model derivation is provided in the supplementary information of \citet{longo2019ed2description}, but for completeness, we provide an abbreviated description below.

\subsubsection{Radiative transfer parameters}

Two-stream radiative transfer theory \citep{meador1980twostream} defines the change in radiative flux through a medium in terms of hemispherically-integrated upward ($F^{\up}_{i}$) and downward ($F^{\down}_{i}$) radiative fluxes via the following system of differential equations (adapting the notation of \citealt{yuan2017reexamination}):

\begin{align}
  & - \frac{dF^{\up}_{i}}{dx} =
  - \underbrace{(a_{i} + \gamma_{i}) F^{\up}_{i}}_{\text{Interception}}
  + \underbrace{\gamma_{i} F^{\down}_{i}}_{\text{Diffuse scatter}}
  + \underbrace{s_{i} F^{\odot}_{i}}_{\text{Direct backscatter}} \label{eqn:twostream-up} \\
  & \frac{dF^{\down}_{i}}{dx} =
  - \underbrace{(a_{i} + \gamma_{i}) F^{\down}_{i}}_{\text{Interception}}
  + \underbrace{\gamma_{i} F^{\up}_{i}}_{\text{Diffuse scatter}}
  + \underbrace{s'_{i} F^{\odot}_{i}}_{\text{Direct scatter}} \label{eqn:twostream-down}
\end{align}

where $dx$ represents the vertical change in the total plant area index (combined area of leaves and woody elements),
$a$ describes absorption of diffuse radiation,
$\gamma$ describes scattering of diffuse radiation,
$s$ and $s'$ describe the upward and downward scattering of direct (``beam'') radiation,
and $F^{\odot}_{i}$ is the incident direct (or ``beam'') radiative flux at canopy layer $i$.

Following \citet{sellers1985canopy}, the coefficients above are defined as follows:

\begin{align}
  & a_{i} + \gamma_{i} = \left[1 - \left( 1 - \beta_{i} \right)\right]\frac{1}{\bar{\mu}_{i}} \\
  & \gamma_{i} = \beta_{i} \omega_{i} \frac{1}{\bar{\mu}_{i}} \\
  & s_{i} = \frac{1}{\mu^{\odot}_{i}} \omega_{i} \beta_{0,i} \\
  & s'_{i} = \frac{1}{\mu^{\odot}_{i}} \omega_{i} \left( 1 - \beta_{0,i} \right)
\end{align}

where
$\bar{\mu}_{i}$ is the optical depth per unit plant area index for diffuse radiation,
$\beta_{i}$ is the backscattering coefficient for diffuse radiation,
$\omega_{i}$ is the scattering coefficient for diffuse radiation,
$\mu^{\odot}_{i}$ is the optical depth per unit plant area index for direct radiation,
and $\beta_{0}$ is the backscattering coefficient for direct radiation.

For a given incident radiation (i.e., solar zenith) angle $\theta$ and leaf orientation angle $\varphi$, the optical depth is defined as:

\begin{equation}\label{eqn:opticaldepth}
  \mu \left( \theta, \varphi \right) = \frac{\cos(\theta)}{G(\theta, \varphi)}
\end{equation}

where $G(\theta, \varphi)$ is a function describing the projected leaf area.
Following \citet{goudriaan_1977_crop}, this function can be approximated as:

\begin{align}
  & G(\theta, \varphi) \approx G^*(\theta, \chi_{i}) = \phi_{1, i} + \phi_{2, i} \cos(\theta) \label{eqn:gfunction} \\
  & \phi_{1, i} = 0.5 - 0.633 \chi_{i} - 0.33 \chi_{i}^{2} \\
  & \phi_{2, i} = 0.877 - \left( 1 - 2 \phi_{1, i} \right)
\end{align}

where $\chi$ is the \emph{leaf orientation factor}---a PFT-specific parameter whose values theoretically range from -1 (vertical leaves) through 0 (randomly-distributed leaf angles) to 1 (horizontal leaves), but in practice are restricted to $-0.4 \leq \chi \leq 0.6$.

For a given solar zenith angle ($\theta_{s}$), the optical depth for direct radiation, $\mu^{\odot}_{i}$, is defined as:

\begin{equation}
  \mu^{\odot}_{i} = \mu(\theta_{s}, \chi_{i}) = \frac{\cos(\theta_{s})}{G^{*}(\theta_{s}, \chi_{i})}
\end{equation}

The optical depth for diffuse radiation, $\bar{\mu}_{i}$, is defined as the integral of \label{eqn:opticaldepth} over all zenith angles:

\begin{equation}
  \bar{\mu}_{i} = \int_{0}^{\frac{\pi}{2}} \frac{\cos(\theta)}{G^{*}(\theta, \chi_{i})} d\theta
  = \frac{1}{\phi_{2,i}} \left[ 1 + \frac{\phi_{1, i}}{\phi_{2, i}} \ln \frac{\phi_{1, i}}{\phi_{1, i} + \phi_{2, i}} \right]
\end{equation}

Following \citet{sellers1985canopy}, diffuse scattering ($\omega$) and backscattering ($\beta$) coefficients of canopy elements (leaves or stems) are defined as a function of those elements' reflectance ($R$) and transmittance ($T$; wood transmittance is assumed to be zero).
(We use index $p$ to refer to PFT and $p(i)$ to refer to the PFT of cohort $i$).

\begin{align}
  & \omega_{i, \leaf} = R_{p(i), \leaf} + T_{p(i), \leaf}\\
  & \omega_{i, \wood} = R_{p(i), \wood}
\end{align}

\begin{align}
  & \beta_{i, \leaf} = \frac{1}{2 \omega_{i}} \left[R_{i, \leaf} + T_{i, \leaf} + \left( R_{i, \leaf} - T_{i, \leaf} \right) J(\chi_{i}) \right] \\
  & \beta_{i, \wood} = \frac{1}{2 \omega_{i}} \left[R_{i, \wood} +  R_{i, \wood} J(\chi_{i}) \right]
\end{align}

where $J(\chi_{i})$ captures the effect of leaf and branch inclination and is approximated as (similarly to \citealt{clm45_note}):

\begin{equation}
  J(\chi_{i}) = \frac{1 + \chi_{i}}{2}
\end{equation}

Both $\omega$ and $\beta$ are calculated independently for leaves and wood and then averaged based on the relative effective area of leaves ($L_{i}$) and wood ($W_{i}$) within a canopy layer.

\begin{align}
  & \omega_{i} = \omega_{i, \leaf} \frac{L_{i}}{L_{i} + W_{i}} + \omega_{i, \wood} (1 - \frac{L_{i}}{L_{i} + W_{i}}) \\
  & \beta_{i} = \beta_{i, \leaf} \frac{L_{i}}{L_{i} + W_{i}} + \beta_{\wood} (1 - \frac{L_{i}}{L_{i} + W_{i}})
\end{align}

To account for non-uniform distribution of leaves within a canopy, ED2 has a PFT-specific \emph{clumping factor} ($q$) parameter that serves as a scaling factor on leaf area index.
Therefore the effective leaf area index ($L$) is related to the true leaf area index ($\LAI$) by:

\begin{equation}
  \label{eqn:elai}
  L_{i} = \LAI_{i} \times q_{p(i)}
\end{equation}

The leaf area of a cohort ($\LAI_{i}$) is calculated as a function of leaf biomass ($B_{\leaf, i}$, \unit{kg C ~ plant^{-1}}), specific leaf area ($\SLA_{p}$, \unit{m^2 ~ kgC^{-1}}), and stem density ($n_{\plant}$, \unit{plants ~ m^{-2}}):

\begin{equation}
  \LAI_{i} = n_{\plant, i} B_{\leaf, i} \SLA_{p(i)}
\end{equation}

In turn, $B_{\leaf, i}$ is calculated from cohort diameter at breast height ($\DBH_{i}$, \unit{cm}) according to the following allometric equations:

\begin{equation}
  B_{\leaf, i} = \bbbl_{p(i)} \DBH_{i}^{\bebl_{p(i)}}
\end{equation}

where $\bbbl_{p(i)}$ and $\bebl_{p(i)}$ are PFT-specific parameters.
The wood area of a cohort ($\WAI_{i}$) is calculated directly from DBH according to a similar allometric equation:

\begin{equation}
  \WAI_{i} = n_{\plant, i} \bbbw_{p(i)} \DBH_{i}^{\bebw_{p(i)}}
\end{equation}

where $\bbbw_{p(i)}$ and $\bebw_{p(i)}$ are PFT-specific parameters.

Backscattering of direct radiation ($\beta^{\odot}_{i}$) is defined as a function of single scattering albedo, ($\alpha_{s}(\theta_{s})$):

\begin{equation}
  \beta^{\odot}_{i} = \frac{\bar{\mu}_{i} + \mu^{\odot}_{i}}{\bar{\mu}_{i}} \alpha_{s}(\theta_{s})
\end{equation}

Single scattering albedo ($\alpha_{s}(\theta_{s})$) is in turn defined as an integral over all illumination angles ($\vartheta$), following \citet{sellers1985canopy} :

\begin{align}
  & \alpha_{s}(\theta_{s}) = \omega \int_{0}^{\frac{\pi}{2}} \frac{\Gamma(\theta_{s}, \vartheta) \cos(\vartheta)} {G(\theta_{s}, \varphi) \cos(\theta_{s}) + G(\vartheta, \varphi) cos(\vartheta)} \sin(\vartheta) d \vartheta \\
  & \Gamma(\theta, \vartheta) = G(\theta, \varphi) G(\vartheta, \varphi) P(\theta, \vartheta) \\
  & \int_{-\frac{\pi}{2}}^{\frac{\pi}{2}} P(\theta, \vartheta) G(\vartheta, \varphi) \sin(\vartheta) d\vartheta = 1
\end{align}

where $P(\theta, \vartheta)$ is the scattering phase function that defines the relative fraction of scattered flux in any direction relative to the projected leaf area in that direction \citep{dickinson1983land}.
Substituting $G = G^{*}$ (equation \ref{eqn:gfunction}),
and assuming uniform scattering (i.e., $P(\theta, \vartheta) = \frac{1}{4\pi}$, and therefore, $\Gamma(\theta, \vartheta) = \frac{G(\theta, \varphi)}{2}$; see detailed discussion of these assumptions in \citealt{yuan2017reexamination})
gives the following analytical solution to this integral:

\begin{equation}
  \alpha_{s,i}(\theta) = \frac{\omega_{i}}{2\left( 1 + \phi_{2,i} \mu^{\odot}_{i} \right)}
  \left[ 1 - \frac{\phi_{1,i}\mu^{\odot}_{i}}{1 + \phi_{2,i}\mu^{\odot}_{i}} \ln\left( \frac{1 + (\phi_{1,i} + \phi_{2,i})\mu^{\odot}_{i}} {\phi_{1,i}\mu^{\odot}_{i}} \right)  \right]
\end{equation}

\subsubsection{Solution for the multi-layer canopy}

In ED2, the direct radiation profile, $F^{\odot}_{i}$, is governed by exponential decay, following Beer's Law:

\begin{align}
  & F^{\odot}_{i} = F^{\odot}_{i+1} \exp\left( -\frac{\TAI_{i}}{\mu^{\odot}_{i}} \right) \\
  & F^{\odot}_{n+1} = F^{\odot}_{\sky}
\end{align}

where $F^{\odot}_{\sky}$ is the incident direct (``beam'') radiation from the atmosphere, a prescribed input;
and $\TAI_{i}$ is the total plant area index, defined as the sum of effective leaf area index ($L_{i}$; equation \ref{eqn:elai}) and wood area index (WAI)

For $n$ cohorts, the full diffuse canopy radiation profile in ED2 is defined by a vector $\vec{A}$ of size $2n + 2$ that contains the upward ($F^{\up}_{i}$) and downward ($F^{\down}_{i}$) radiative fluxes for every ``interface'' \emph{immediately below cohort $i$}; therefore,
$F^{\up}_{(n+1)}$ refers to the upward diffuse radiative flux from the top of the canopy towards the atmosphere (the quantity used to calculate the albedo),
$F^{\down}_{(n+1)}$ refers to the downward diffuse radiative flux from the atmosphere into the top of the canopy,
$F^{\up}_{1}$ refers to the upward diffuse radiative flux from the ground into the canopy layer of the shortest cohort,
and
$F^{\down}_{1}$ refers to the downard diffuse radiative flux from the canopy layer of the shortest cohort towards the ground.
To derive each of these $F$ terms, ED2 uses the following analytical solution for equations \ref{eqn:twostream-down} and \ref{eqn:twostream-up} (see \citealt{longo2019ed2description}, Section S12, for a full derivation):

\begin{align}
  & F^{\down}_{i}
    = x_{(2i-1)}\gamma^{+}_{i}  \exp\left( -\lambda_{i} \TAI \right)
    + x_{(2i)}\gamma^{-}_{i}    \exp\left( +\lambda_{i} \TAI \right)
    + \delta^{+}               \exp\left( -\frac{\TAI_{i}}{\mu^{\odot}_{i}} \right) \label{eqn:edr-fdown} \\
  & F^{\up}_{i}
    = x_{(2i-1)}\gamma^{-}_{i}  \exp\left( -\lambda_{i} \TAI \right)
    + x_{(2i)}\gamma^{+}_{i}    \exp\left( +\lambda_{i} \TAI \right)
    + \delta^{-}               \exp\left( -\frac{\TAI_{i}}{\mu^{\odot}_{i}} \right) \label{eqn:edr-fup}
\end{align}

where $x_{i}$ is a vector of wavelength- and cohort-specific unknowns, and the remaining terms are:

\begin{align}
  & \gamma^{\pm}_{i} = \frac{1}{2} \left( 1 \pm \sqrt{ \frac{1-\omega_{i}}{1-(1-2\beta_{i})\omega_{i}} } \right) \\
  & \delta^{\pm}_{i} = \frac{\left( \kappa^{+} \pm \kappa^{-} \right) \mu^{\odot 2}_{i}}{2\left( 1 - \lambda_{i}^{2} \mu^{\odot 2}_{i}\right)} \\
  & \lambda^{2}_{i} = \frac{\left[ 1-(1-2\beta_{i})\omega_{i} \right]\left(1-\omega_{i}\right)}{\bar{\mu}_{i}^{2}} \\
  & \kappa^{+}_{i} = -\left[\frac{1 - \left(1 - 2\beta_{i}\right)\omega_{i}}{\bar{\mu}_{i}} + \frac{1 - 2\beta_{i}}{\mu^{\odot}_{i}}\right] \frac{\omega_{i}F^{\odot}_{(i+1)}}{\mu^{\odot}_{k}} \\
  & \kappa^{-}_{i} = -\left[\frac{\left(1 - 2\beta_{i}\right)\left(1-\omega_{i}\right)}{\bar{\mu}_{i}} + \frac{1}{\mu^{\odot}_{i}}\right] \frac{\omega_{i}F^{\odot}_{(i+1)}}{\mu^{\odot}_{k}}
\end{align}

The problem of solving for $x_{i}$ in equations \ref{eqn:edr-fdown} and \ref{eqn:edr-fup} can be written as a matrix equation:

\begin{equation}
  \mat{S} \vec{x} = \vec{Y}
\end{equation}

where $\vec{x} = \left( x_{1},x_{2}, \ldots, x_{2n+1}, x_{2n+2} \right)$,
$\vec{y} = \left( y_{1}, y_{2}, \ldots, y_{(2n+1)}, y_{(2n+2)} \right)$,
and $\mat{S}$ is a $(2n + 2) \times (2n + 2)$ tridiagonal matrix.
To solve this matrix equation, ED2 defines the following boundary conditions:
At the top of the canopy ($i=n+1$),
$F^{\down}_{(n+1)} \equiv F^{\down}_{\sky}$, the incident diffuse flux from the atmosphere (a prescribed input);
$\TAI_{(n+1)} = 0$;
$\bar{\mu}_{(n+1)} = 1$;
$\omega_{(n+1)} = 1$ (no absorption);
and
$\beta_{(n+1)} = \beta^{\odot}_{(n+1} = 0$ (no scattering; all radiance is transmitted).
At the bottom of the canopy ($i=1$), we re-define the ground scattering, $\omega_{g}$, based on a soil radiative transfer model (Section \ref{subsec:edr-prospect}).
With these boundary conditions, the elements of $\mat{S}$ are given by:

\begin{align}
  \begin{split}
    S_{1,1}       &= \left( \gamma^{-}_{1} - \omega_{g} \gamma^{+} \right) \exp(-\lambda_{1} \TAI_{1}) \\
    S_{1,2}       &= \left( \gamma^{+}_{1} - \omega_{g} \gamma^{-} \right) \exp(+\lambda_{1} \TAI_{1}) \\
    S_{(2i, 2i-1)} &= \gamma^{+}_{i} \\
    S_{(2i, 2i)}  &= \gamma^{-}_{i} \\
    S_{(2i, 2i+1)} &= - \gamma^{+}_{(i+1)} \exp(-\lambda_{(i+1)} \TAI_{(i+1)}) \\
    S_{(2i, 2i+2)} &= - \gamma^{-}_{(i+1)} \exp(+\lambda_{(i+1)} \TAI_{(i+1)}) \\
    S_{(2i+1, 2i-1)}  &= \gamma^{-}_{i} \\
    S_{(2i+1, 2i)}    &= \gamma^{+}_{i} \\
    S_{(2n, 2n+1)} &= - \gamma^{+}_{(n+1)} \exp(-\lambda_{(n+1)} \TAI_{(n+1)}) \\
    S_{(2n, 2n+2)} &= - \gamma^{-}_{(n+1)} \exp(+\lambda_{(n+1)} \TAI_{(n+1)}) \\
  \end{split}
\end{align}

and the elements of $\vec{y}$ are given by:
  
\begin{align}
  \begin{split}
    y_{1}         &= \omega_{0} F^{\odot}_{1} - \left(\delta^{-}_{1} - \omega_{g} \delta^{+}_{1}\right) \exp\left( - \frac{\TAI}{\mu^{\odot}_{1}} \right) \\
    y_{(2i)}      &= \delta^{+}_{(i+1)} \exp\left(- \frac{\TAI_{(i+1)}}{\mu^{\odot}_{(i+1)}} \right) - \delta^{+}_{i} \\
    y_{(2i+1)}    &= \delta^{-}_{(i+1)} \exp\left(- \frac{\TAI_{(i+1)}}{\mu^{\odot}_{(i+1)}} \right) - \delta^{-}_{i} \\
    y_{(2n+2)}    &= F^{\down}_{\sky} - \delta^{+}_{(n+1)} \\
  \end{split}
\end{align}

Finally, the surface albedo ($\rho$) is defined as the fraction of the total radiative flux incident on the canopy ($F^{\odot}_{\sky} + F^{\down}_{\sky}$) that is reflected:

\begin{equation}
  \rho = \frac{F^{\up}_{(n+1)}}{F^{\odot}_{\sky} + F^{\down}_{\sky}}
\end{equation}

\subsection{ED2-PROSPECT coupling}\label{subsec:edr-prospect}

By default, ED2 performs the canopy shortwave radiative transfer calculations described in two broad spectral bands:
visible (400--700 \unit{nm}) and near-infrared (700--2500 \unit{nm}).
For each of these regions, ED2 has user-defined prescribed, PFT-specific leaf and wood reflectance and transmittance values, and calculates soil reflectance as the average of constant wet and dry soil reflectance values weighted by the relative soil moisture (0 = fully dry, 1 = fully wet).
In this study, we modified ED2 to perform the same canopy radiative transfer calculations but in 1 \unit{nm} increments across the range 400--2500 \unit{nm}.
We then simulated leaf reflectance and transmittance using the PROSPECT 5 leaf RTM,
which has the following five parameters:
Effective number of leaf mesophyll layers (\emph{N}, unitless, >= 1),
total chlorophyll content (\emph{Cab}, \unit{\mu g ~ cm^{-2}}),
total carotenoid content (\emph{Car}, \unit{\mu g ~ cm^{-2}}),
water content (\emph{Cw}, \unit{g ~ cm^{-2}}),
and dry matter content (\emph{Cm}, \unit{g ~ cm^{-2}})
\citep{feret2008prospect4}.
For wood reflectance, we used a single representative spectrum---the mean of all wood spectra from \citet{asner1998biophysical}, resampled to 1 \unit{nm} resolution---for all PFTs.
For soil scattering ($\omega_{g}$), we used the simple Hapke soil submodel used in the Soil-Leaf-Canopy RTM \citep{verhoef2007coupled}, whereby soil reflectance is the average of prescribed wet and dry soil reflectance spectra weighted by a relative soil moisture parameter ($\varrho_{\soil}$, unitless, 0--1).
The final coupled PROSPECT-ED2 canopy radiative transfer model (hereafter known as ``EDR'') has 12 parameters for each PFT---
5 parameters for PROSPECT, specific leaf area, two parameters each for the leaf and wood allometries, and clumping ($q$) and leaf orientation ($\chi$) factors---and one site-specific parameter---the relative soil moisture (Table \ref{tab:parameters}).

\subsection{Site and data description}\label{subsec:site-data}

For model calibration, we selected 54 sites from the NASA Forest Functional Types (FFT) field campaign that contained plot-level inventory data coincident with observations of the NASA Airborne Visible/Infrared Imaging Spectrometer-Classic (AVIRIS-Classic).
A full description of this dataset is provided in \citet{singh2015imaging}.
Briefly, each site consisted of a 60 x 60 m transect within which forest inventory data (stem density, species identity, and diameter at breast height, DBH) were collected.
These sites are located in the United States Upper Midwest, northern New York, and western Maryland (Figure~\ref{fig:site-map}),
and include stands dominated by either evergreen or deciduous trees and spanning a wide range of structures, from dense groups of saplings to sparse groups of large trees (Figure~\ref{fig:site-structure}).

We grouped the tree species in these sites into five different PFTs as defined by ED2:
Early successional hardwood, northern mid-successional hardwood, late successional hardwood, northern pine, and late successional conifer.
The mappings of tree species onto these PFTs are provided as a CSV-formatted table in the file \texttt{inst/pfts-species.csv} in the source code repository for this project (see Code and Data Availability section).

\begin{figure}
  \centering
  \includegraphics[width=\textwidth]{figures/site-map}
  \caption{\
    Map of sites used in this analysis.
    Sites shown in in Figure~\ref{fig:spec-error-all} are labeled.
  }\label{fig:site-map}
\end{figure}

\begin{figure}
  \centering
  \includegraphics[width=\textwidth]{figures/site-structure}
  \caption{\
    Stand structure and composition characteristics of sites selected for analysis.
    The dashed line is a forest self-thinning curve \citep[c.f.,][]{zeide2010comparison} parameterized based on an analysis of US Forest Service Forest Inventory and Analysis (FIA) data (T.\ Andrews, \emph{unpublished}).
  }\label{fig:site-structure}
\end{figure}

AVIRIS-Classic measures the directional radiance of a surface from 365 to 2500 \unit{nm} at approximately 10 \unit{nm} increments.
Atmospheric correction routines use this level 1 radiance product to estimate the surface reflectance (technically, hemispherical-directional reflectance factor, HDRF, sensu \citealt{schaepman-strub2006reflectance})---a quantity that is (in theory) independent of illumination conditions and therefore can be more directly related to intrinsic physical properties of the surface.
For this study, in addition to the standard atmospheric correction and orthorectification conducted by NASA JPL,
the AVIRIS data were also cross-track illumination corrected and BRDF-corrected, following the procedure of \citet{lucht2000algorithm}.
Briefly, this BRDF correction estimates ``intrinsic surface albedo''---the quantity that is simulated by EDR---from directional reflectance data through application of a polynomial approximation to the Ross-Li semiempirical BRDF model.
The full AVIRIS processing pipeline for the  AVIRIS data (including the BRDF approximation) we used is described in \citet{singh2015imaging}.

Because of unrealistic values in the shortwave infrared spectral region (>1300 \unit{nm}) in the AVIRIS observations (likely caused by faulty atmospheric correction), we only used observations from 400 to 1300 \unit{nm} for model calibration and validation.
Following \citet{shiklomanov2016quantifying}, we used the relative spectral response functions of AVIRIS-Classic to relate the 1 \unit{nm} EDR predictions to the 10 \unit{nm} AVIRIS-Classic measurements.

For each AVIRIS-Classic observation, we retrieved the solar zenith angle directly from the flightline metadata where available, and calculated it based on the local time and position if not.
In addition, we retrieved the relative fraction of diffuse vs.\ direct incident radiation from the hourly MERRA-2 meteorological reanalysis (GMAO 2015) for each observation's location and time (rounded to the nearest hour).

\subsection{Model calibration}

To estimate EDR parameters from AVIRIS observations, we used a Bayesian approach that builds on our previous work at the leaf scale \citep{shiklomanov2016quantifying}.
For a parameter vector $\vec{\Theta}$ and matrix of observations $\mat{X}$, the typical form of Bayes' rule is given by:

\begin{equation}
  \underbrace{P(\vec{\Theta} | \mat{X})}_{\text{Posterior}} \sim \underbrace{P(\mat{X} | \vec{\Theta})}_{\text{Likelihood}} \underbrace{P(\vec{\Theta})}_{\text{Prior}}
\end{equation}

Rather than performing a separate calibration at each site, we performed a single joint calibration across all sites.
Therefore, our overall likelihood ($P(\mat{X} | \vec{\Theta})$) was the product of the likelihood at each site ($P(\mat{X}_{s} | \vec{\Theta})$, for site $s$):

\begin{equation}
  P(\mat{X} | \vec{\Theta}) = \prod_{s} P(\mat{X}_{s} | \vec{\Theta})
\end{equation}

The likelihood at each site $s$ is based on how well EDR predicted albedo ($R_{\mathrm{pred}, s}$) matches that site's observed AVIRIS albedo ($\mat{X}_{s}$)
given the known forest composition at that site ($\mathrm{comp}_{s}$) and the current estimate of the overall parameter vector.
Similar to \citet{shiklomanov2016quantifying}, we assumed the residual error between predicted and observed reflectance followed a multivariate normal distribution ($\mathrm{MvNormal}$):

\begin{equation}
  P(\mat{X}_{s} | \vec{\Theta}) =
  \mathrm{MvNormal}(\mat{X}_{s} | R_{\mathrm{pred}, s}, \mat{\Sigma_{s}})
\end{equation}

where $\Sigma$ is the residual variance-covariance matrix.
\citet{shiklomanov2016quantifying} assumed $\mat{\Sigma}$ was a diagonal matrix with the same residual variance for all elements.
For this study, we made two important changes to this methodology:
First, to account for the large differences in the range of feasible reflectance values in different wavelength regions
(for vegetation, reflectance in the 400--700 \unit{nm} range is typically much lower than in the 700--1400 \unit{nm} range),
we used a heteroskedastic error model where the residual standard deviation ($\vec{\sigma}_{s}$) was a linear function of the predicted reflectance $R_{(\mathrm{pred}, s)}$ with slope $m$ and intercept $b$.
Second, to account for autocorrelation in hyperspectral bands, we replaced the diagonal residual covariance matrix with an order-1 autoregressive (AR-1) covariance matrix.
Collectively, these two changes produce the following calculation for $\Sigma$:

\begin{align}
  & \vec{\sigma_{s}} = m R_{\mathrm{pred}, s} + b \\
  & \mat{\Sigma_{s}} = \vec{\sigma_{s}} \varrho^{\mat{H}} \vec{\sigma_{s}}
\end{align}

where $H$ is a matrix describing the distance between bands (0 on the diagonal, increasing regularly toward the corners) and $\varrho$ is the AR-1 autocorrelation parameter.
To simplify the inversion procedure, we first performed the inversion using a diagonal covariance matrix (i.e., $\varrho = 0$), then calculated the mean $\varrho$ from the residuals of this fit, and then used this average value (0.700) for our final inversion.

% TODO: Nope! Uniform for LAI 0-10
In addition, to mitigate sampling issues related to EDR's saturating response to increasing total LAI, we added an additional term to our likelihood that assigns a fixed lognormal probability distribution (with parameters 1 and 0.5, respectively) to the EDR predicted LAI for a given site ($\LAI_{\mathrm{pred}, s}$).
The parameters 1 and 0.5 produce a distribution with 95\% of the density between 1.0 and 7.2, which spans the range of LAI values typically observed temperate forests.
The final expression for the site-specific likelihood is therefore:

\begin{equation}
  R_{\mathrm{pred}, s}, \LAI_{\mathrm{pred}, s} = \EDR(\vec{\Theta} | \mathrm{comp}_{s})
\end{equation}

\begin{equation}
  P(\mat{X}_{s} | \vec{\Theta}) =
  \mathrm{MvNormal}(\mat{X}_{s} | R_{\mathrm{pred}, s}, \mat{\Sigma}) ~
  \mathrm{LogNormal}(\LAI_{\mathrm{pred}, s} | 1, 0.5)
\end{equation}

Therefore, our parameter vector $\vec{\Theta}$ consists of the following (summarized in Table~\ref{tab:parameters}):
10 EDR parameters per PFT---5 parameters for the PROSPECT 5 model (\emph{N}, \emph{Cab}, \emph{Car}, \emph{Cw}, \emph{Cm}) and 5 EDR parameters related to canopy structure ($q$, $\chi$, $\SLA$, $\bbbl$, $\bbbw$)---,
1 parameter per site (relative soil moisture, $\psi_{s}$),
and the residual slope ($m$) and intercept ($b$).
(With 5 PFTs and 54 sites, this means that $\vec{\Theta}$ has length $(10 \times 5) + 54 + 2 = 106$).

For priors on the PROSPECT 5 parameters and SLA, we performed a hierarchical multivariate analysis \citep{shiklomanov2020does} on PROSPECT 5 parameters and direct SLA measurements from \citep[][Chapter 3]{shiklomanov_dissertation}.
For priors on the leaf biomass allometry parameters, we fit a multivariate normal distribution to allometry coefficients from \citet{jenkins2003nationalscale,jenkins2004comprehensive} using the \texttt{PEcAn.allometry} package (\url{https://github.com/pecanproject/pecan/tree/develop/modules/allometry}).
For the clumping factor, we used a uniform prior across its full range (0 to 1), and for the leaf orientation factor, we used a weakly informative beta distribution re-scaled to the range $(-1, 1)$ and centered on 0.5 (Table~\ref{tab:parameters}).

To alleviate issues with strong collinearity between the allometry parameters and the specific leaf area, we fixed the allometry exponent parameters ($\bebl$ and $\bebw$) to their prior means for each PFT.
Doing so dramatically improved the stability of the inversion algorithm and the accuracy of the results.

We fit this model using the Differential Evolution with Snooker Update (``DEzs'') Markov-Chain Monte Carlo (MCMC) sampling algorithm \citep{terbraak2008differential} as implemented in the R package \texttt{BayesianTools} \citep{bayesiantools}.
We ran the algorithm using 8 independent chains for as many iterations as required to achieve convergence, assessed according to a Gelman-Rubin Potential Scale Reduction Factor (PSRF) diagnostic value of less than 1.1 for all parameters \citep{gelman1992inference}.

\begin{table}
  \caption{EDR parameters and prior distributions}
  \label{tab:parameters}
  \begin{tabular}{llp{2.2in}ll}
    \tophline
    Type & Name & Description & Unit & Prior \\
    \middlehline
    \multirow[t]{5}{1.4in}{\parbox[t]{1.4in}{Leaf RTM parameters\\(1 per PFT)}}
    & \emph{N} & Effective number of leaf mesophyll layers & unitless & $\mathrm{MvNormal}(\vec{\mu}, \mat{\Sigma})$$^1$ \\
    & \emph{Cab} & Total leaf chlorophyll content & \unit{\mu g ~ cm^-2} & $\mathrm{MvNormal}(\vec{\mu}, \mat{\Sigma})$$^1$ \\
    & \emph{Car} & Total leaf carotenoid content & \unit{\mu g ~ cm^-2} & $\mathrm{MvNormal}(\vec{\mu}, \mat{\Sigma})$$^1$ \\
    & \emph{Cw} & Leaf water content & \unit{g ~ cm^-2} & $\mathrm{MvNormal}(\vec{\mu}, \mat{\Sigma})$$^1$ \\
    & \emph{Cm} & Leaf dry matter content & \unit{g ~ cm^-2} & $\mathrm{MvNormal}(\vec{\mu}, \mat{\Sigma})$$^1$ \\
    \multirow[t]{5}{1.4in}{\parbox[t]{1.4in}{Canopy RTM parameters\\(1 per PFT)}}
    & $\SLA$ & Specific leaf area & \unit{kg ~ m^-2} & $\mathrm{MvNormal}(\vec{\mu}, \mat{\Sigma})$$^1$ \\
    & $q$ & Canopy clumping factor & unitless & $\mathrm{Uniform}(0, 1)$ \\
    & $\chi$ & Leaf orientation factor & unitless & $2 \times \mathrm{Beta}(18, 12) - 1$ \\
    & $\bbbl$ & Leaf biomass allometry base & unitless & $\mathrm{LogNormal}(m_{l}, s_{l})$$^2$ \\
    & $\bbbw$ & Wood biomass allometry base & unitless & $\mathrm{LogNormal}(m_{w}, s_{w})$$^2$\\
    \multirow[t]{3}{1.4in}{Other parameters}
    & $\psi_{s}$ & Relative soil moisture content at site $s$ & unitless & $\mathrm{Uniform}(0, 1)$ \\
    & $a$ & Residual slope & unitless & $\mathrm{Exponential}(1)$ \\
    & $b$ & Residual intercept & unitless & $\mathrm{Exponential}(10)$ \\
    \bottomhline
  \end{tabular}
  \belowtable{
    $^1$ PFT-specific multivariate normal distribution fit to PROSPECT parameters and SLA from Shiklomanov (2018), chapter 3.\\
    $^2$ PFT-specific results from Bayesian fits of allometric equations to allometry data from \citet{jenkins2003nationalscale,jenkins2004comprehensive} using the \texttt{PEcAn.allometry} package.
  }
\end{table}

\subsection{Analysis}

To assess the extent to which AVIRIS-Classic observations were able to constrain parameter estimates, we compared the prior and posterior distributions for all parameters.
To evaluate the performance of the calibrated model, we compared the posterior credible and predicted 95\% intervals of EDR-predicted spectra against the AVIRIS observations at each site.
We also compared the EDR-predicted LAI against field observations at each site.
To evaluate goodness-of-fit and additive and multiplicative biases, we used an ordinary least squares regression of mean observed vs.\ posterior mean predicted LAI.
