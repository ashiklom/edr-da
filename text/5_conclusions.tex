\section{Conclusions}

The objective of this study was to calibrate the canopy radiative transfer scheme inside the ED2 dynamic vegetation model by comparing its predictions of surface reflectance against airborne imaging spectroscopy data.
The calibration successfully constrained the posterior distributions of model parameters related to canopy structure (leaf angle, canopy clumping, and leaf area index) for five plant functional types characteristic of temperate forests of the northeastern United States.  
However, comparisons of predicted spectra post-calibration against observations reveal widespread biases.
This suggests that there are structural issues with the ED2 radiative transfer model that inhibit its ability to accurately predict surface optical properties.
Sensitivity analyses, along with comparison against an alternative canopy radiative transfer model more commonly used by the remote sensing community (4SAIL), shed additional light on the problem and provides avenues for future exploration and model improvement.
One issue was unrealistically high sensitivity to wood reflectance, which could be addressed by calibration of parameters related to wood area index (such as wood allometries) or, if that fails, alternative representations of the contribution of wood reflectance to canopy reflectance.
That being said, wood reflectance alone was insufficient to explain bias in predicted spectra.\@
We suggest that this error is likely related to soil reflectance, but additional sensitivity analyses (for instance, by varying soil reflectance in simulations with dense, closed canopies) are required to confirm this.
Ultimately, this work demonstrates the utility of using surface reflectance predictions to evaluate model representations of canopy radiative transfer, and thus contributes to the rapidly expanding body of literature on applications of remote sensing to dynamic vegetation modeling.
