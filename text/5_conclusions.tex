\conclusions  %% \conclusiosn[modified heading if necessary]

Remote sensing observations are unrivaled in their spatial completeness and extent, notably extending to regions like the tropics and high latitudes that are relatively undersampled but have a disproportionate impact on the global climate system~\citep{schimel2015observing} and/or global biodiversity~\citep{jetz_2016_monitoring}.
At the same time, satellite time series provide multidecadal records with relatively high temporal frequency, which have tremendous utility for calibrating model projections of past ecological dynamics~\citep{kennedy2014bringing, pasquarella2016imagery}.
Used in combination with other emerging data sources, including global trait databases and eddy covariance measurements, remote sensing can be a transformative force in ecosystem ecology.

In this paper, we showed that using a vegetation model to directly simulate surface reflectance is a promising approach for calibrating and validating models against remotely sensed observations.
To do this, we modified the ED2 dynamic vegetation model to predict full-range hyperspectral surface reflectance and then calibrated this modified model against airborne imaging spectroscopy data.
The calibration successfully constrained the distributions of model parameters related to canopy structure and leaf biogeochemistry for five plant functional types for five plant functional types characteristic of temperate forests of the northeastern United States.
The calibrated model was able to accurately reproduce surface reflectance and leaf area index for sites with highly varied forest composition and structure, using a single common set of parameters (i.e., not site-specific parameters).
Although our study focused only on the ED2 model, the basic structure and assumptions of the ED2 canopy radiative transfer scheme are shared by many other vegetation models, so we expect that our approach has high transferability within the vegetation modeling community.
