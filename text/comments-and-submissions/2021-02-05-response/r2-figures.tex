\documentclass{article}
\usepackage{graphicx}

\title{Supplementary figures for reviewer T. Quaife}

\begin{document}

\maketitle

\begin{figure}[ht]
  \centering
  \includegraphics[width=\textwidth]{figures/edr-sail-comparison-lai.png}
  \caption{Comparison of EDR and PRO4SAIL simulations for single-cohort canopies over a range of leaf area indices (LAI). These simulations assume common PROSPECT parameters, and sun-sensor geometries (nadir sensor, cosine solar zenith angle of 0.85). For EDR, we use a single cohort with the prescribed LAI. “SAIL: HR” is the “hemispherical reflectance” (“blue-sky albedo”), calculated (as in EDR) as the average of directional-hemispherical (DHR) and bi-hemispherical (BHR) reflectance streams weighted by the direct sky fraction (0.9; same value as in EDR). Similarly, “SAIL: DR” is the “directional reflectance”, calculated analogously from the bi-directional (BDR) and hemispherical-directional (HDR) reflectance streams. Individual SAIL fluxes are shown with dotted lines. Here, cos(solar zenith angle) is 0.85 -- a typical value for summer in our study domain.
}
\end{figure}

\begin{figure}[ht]
  \centering
  \includegraphics[width=\textwidth]{figures/edr-sail-comparison-czen.png}
  \caption{(New supplementary figure) Same as above, but over a range of solar geometries (czen = cosine of solar zenith angle). All simulations use a common LAI of 3.}
\end{figure}

\end{document}
