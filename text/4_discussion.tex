\section{Discussion}

Calibrating and validating vegetation models using optical remote sensing data typically involves derived data products (e.g., MODIS GPP) that rely on their own models;
in other words, ``bringing the observations closer to the models''.
In this study, we presented an alternative approach whereby we bring the models closer to the observations by training a vegetation model to simulate full-range hyperspectral surface albedo as observed by optical remote sensing instruments.
We then demonstrated how this approach could be used to calibrate the model against airborne imaging spectroscopy data from AVIRIS-Classic.
We found that such calibration reduced uncertainties in parameters related to leaf biochemistry and canopy structure, even for parameters with well-informed priors (Figure~\ref{fig:posterior-pft}).
Moreover, we found that that the calibrated model was able to reproduce observed surface albedo (Figures~\ref{fig:spec-error-all} and~\ref{fig:spec-error-aggregate}) and leaf area index (Figure~\ref{fig:lai-pred-obs}) reasonably well across large number of structurally, compositionally, and geographically diverse sites (Figure~\ref{fig:sites}).

Compared to previous similar efforts that have coupled vegetation models to external canopy radiative transfer models~\citep{knorr2001assimilation, nouvellon2001coupling, quaife2008assimilating},
our work is novel because it uses a canopy radiative transfer formulation that \emph{already exists inside the model itself}.
This reduces the number of new assumptions and variables we have to introduce and increases the extent to which constraint on canopy radiative transfer parameters propagates to other related processes in the model.
More importantly, in such a coupling, the only way that observed reflectance constrains the model is through the foliar biomass, and additional information from the reflectance on canopy structure is confined to the GORT parameters.
By contrast, in our approach, parameters and states in the shortwave canopy radiative transfer submodel also influence other model processes, including thermal radiative transfer, micrometeorology, and competition~\citep{longo2019ed2description}, with profound consequences for model predictions of ecosystem fluxes and composition~\citep{viskari_2019_influence}.

The canopy radiative transfer model in ED2 is derived from the two-stream model of \citet{sellers1985canopy} and adapted to a multi-level canopy.
Similar versions of this two-stream formulation are present in other land surface models, including CLM~\citep{clm45_note}, SiB~\citep{baker2008seasonal}, Noah~\citep{niu2011community}, tRIBS-VEGGIE~\citep{ivanov2008vegetationhydrology}, IMOGEN~\citep{huntingford2008quantifying}, and JULES~\citep{best_2011_joint}.
Although the exact parameterization and implementation differs across these models, the similarity of the underlying conceptual framework means that our approach should be directly transferable to all of these models as well.

Nevertheless, our analysis echoed some known challenges in canopy radiative transfer modeling.
One challenge is equifinality in the contributions of leaf biochemistry, leaf morphology, and different aspects of canopy structure to canopy albedo, which means that multiple variable and parameter combinations can produce very similar canopy albedo responses (\citealt{lewis2007spectral}; Figures TODO).
We mitigated the equifinality between leaf traits and canopy structure by using informative priors on leaf traits from an independent data source~\citep{shiklomanov_dissertation}.
However, there is additional equifinality in the effects of the EDR canopy structure parameters.
For example, because the effective LAI used in EDR’s actual radiative transfer calculations is defined as the product of ``true'' LAI and clumping factor (equation~\ref{eqn:elai}), and because LAI is, in turn, derived from multiple parameters (leaf biomass allometry, specific leaf area; equation~\ref{eqn:lai}), these parameters collectively cannot be independently determined from reflectance data alone.
At the same time, increasing the leaf orientation factor (more horizontal, or ``planophile'', leaf orientation) has a similar (although not identical) effect to increasing LAI and clumping factor---namely, increasing canopy reflectance, especially in the near-infrared (Figure TODO).
Collectively, these issues may help explain some of the edge-hitting behavior (parameter distributions clustered at the ends of the distribution) observed in our posterior estimates (Figure~\ref{fig:posterior-pft}), and some of the bias in our LAI estimates (Figure~\ref{fig:lai-pred-obs}).

That being said, one major advantage of the Bayesian calibration approach is that its output is a joint posterior distribution that includes not only fully quantified uncertainties for each parameter but also the variance-covariance matrix of each parameter.
Equifinality in parameters would manifest as strong pairwise correlation between parameters in the posterior distribution.
Examining this correlation matrix (Figure TODO) shows that there are some parameter pairs with strong correlations, such as the positive correlations between leaf and wood allometries for all PFTs except northern pine, and the hypothesized negative correlation between the leaf allometry (LAI) and clumping factor, which was only observed for the early-successional hardwoods and northern pines.
However, these correlations mostly do not occur in the parameters that exhibited strong edge-hitting behavior---namely, clumping and orientation factors for mid- and late-successional hardwood PFTs (Figure~\ref{fig:posterior-pft}).
Strong correlations also occurred among some of the PROSPECT parameters, and between PROSPECT and structural parameters, but contributed little to equifinality because the strong constraints on PROSPECT led to overall small covariance terms.
Finally, because our calibration captured all of these covariances the presence of moderate equifinality did not preclude ecologically meaningful parameter constraints or accurate predictions because these covariances are being propagated into predictions.
This is directly analogous to how a linear regression can have a tight confidence interval, despite high correlations between the slope and intercept, with that equifinality driving the characteristic hourglass shape of a regression confidence interval.

EDR consistently predicted lower hemispherical reflectance than SAIL (Figure TODO).
This difference can be attributed primarily to differences in how each model defines the direct radiation backscatter coefficient in the radiative transfer equation.
A detailed description of the discrepancy is provided in~\citet{yuan2017reexamination}.
Briefly, EDR (and the Sellers 1985 model from which EDR is derived) defines direct radiation backscatter as a function of the single-scattering albedo (equation~\ref{eqn:backscatter-direct}), which in turn is a challenging integral involving the leaf scattering phase function and leaf projected area function (equation~\ref{eqn:ssa-integral}).
The analytical solution to this integral in EDR (equation~\ref{eqn:ssa-solved}) assumes a uniform scattering phase function, which is appropriate for point scatterers but less so for horizontal surfaces like leaves.
The practical consequence of this assumption is a lower value of the direct radiation backscatter and therefore a lower albedo, which is consistent with the results of our sensitivity analysis.
This underestimation of albedo described above may also help explain the edge-hitting behavior in our posterior distributions (Figure~\ref{fig:posterior-pft}) as well as the relatively low accuracy of our LAI estimates (Figure~\ref{fig:lai-pred-obs}).
Specifically, our EDR calibration may be trying to compensate for underestimated albedos via a tendency for prefer higher effective LAI values (which results in higher values of the leaf biomass allometry and clumping factor for some PFTs) and more horizontal leaf distributions (i.e., higher leaf orientation factor), both of which increase albedo (Figure TODO).

Meanwhile, the SAIL definition of direct backscatter is a more simple function of leaf scattering, leaf angle distribution, and canopy optical depth that also produces a more accurate albedo estimate~\citep{yuan2017reexamination}.
Given that underestimating albedo can have significant consequences for the biological, ecological, and physical predictions of ED2~\citep{viskari_2019_influence}, incorporating this fix into the ED2 canopy RTM is an important future direction of our work.
However, doing so is outside the scope of this work because doing so would require propagating the different coefficients through the complex, multiple-canopy-layer solution of EDR (Section \ref{subsubsec:multi-canopy-solution})---a non-trivial task.

A significant caveat to the broader application of our approach is that there is a subtle but significant difference between the physical quantity EDR predicts and the quantities typically measured by optical remote sensing.
Specifically, EDR predicts the hemispherical reflectance---the ratio of total radiation leaving the surface to the total radiation incident upon the surface, integrated over all viewing angles (a.k.a, ``blue-sky albedo'').
On the other hand, optical remote sensing platforms typically measure the directional reflectance factor---the ratio of actual radiation reflected by a surface to the radiation reflected from an ideal diffuse scattering source (e.g., a Spectralon calibration panel) subject to the same illumination, in a specific viewing direction~\citep{schaepman-strub2006reflectance}.
These two quantities are numerically equivalent only for ideal Lambertian surfaces or, for non-Lambertian surfaces, under specific sun-sensor geometries.
However, vegetation canopies---the focus of this study---are known to exhibit reflectance with very strong angular dependence, so a comparison of canopy hemispherical reflectance with a remotely sensed directional reflectance factor is invalid.

Our specific analysis is valid because---as described in the Methods---we used AVIRIS data that were also BRDF-corrected, whereby the directional reflectance estimates from the atmospheric correction process were further converted to estimates of hemispherical reflectance via a polynomial approximation of the Ross-Li semiempirical BRDF model~\citep{lucht2000algorithm}.
Another dataset that would have been valid for our analysis (albeit, one with much lower spatial and spectral resolution) is the MODIS albedo product (MOD43), which takes advantage of the angular sampling of the MODIS instrument to quantify the surface BRDF characteristics and therefore more precisely estimate the albedo~\citep{wang2004using, schaaf2015mcd43a1}.
However, our approach as described here would \emph{not} be valid for the surface reflectance products produced by nadir-viewing instruments such as Landsat, Sentinel-2, or most airborne platforms, at least without additional processing steps on the data,
or, preferably, modification of the underlying radiative transfer models to allow for the prediction of directional reflectance.
Fortunately, the assumptions and parameters that comprise two-stream radiative transfer models like EDR and its parent model~\citep{sellers1985canopy} are readily adaptable to prediction of directional reflectance.
For example, the SAIL model~\citep{verhoef1984light, verhoef2007coupled}---which predicts both hemispherical and directional reflectance, and which has a long history of successful application to remote sensing---makes the same general assumptions as EDR and even shares many of the underlying coefficients~\citep{yuan2017reexamination}.
Alternatively, land surface models can take advantage of recent advances in radiative transfer theory to improve their accuracy without significant computational penalty~\citep[e.g.,][]{hogan_2018_fast}.

A related issue is the missing or simplistic treatment of two- and three-dimensional heterogeneity in canopy structure in EDR.\@
For one, the treatment of leaves as infinitely small elements randomly distributed through the canopy space---a common feature of all two-stream approximations---neglects complex realities of the canopy light environment such as gaps and self-shading.
In EDR, self-shading is handled via the clumping factor parameter, which functions as a scalar correction on the leaf area index (Equation~\ref{eqn:elai}).
A key feature of EDR design is its representation of multiple co-existing plant cohorts competing for light within a single patch;
however, horizontal heterogeneity and competition between these cohorts is ignored.
Improved representation of lateral energy transfer can improve the accuracy of simulations of the canopy light environment, and recent theoretical advances show that this can be accomplished without a significant loss in computational performance~\citep{hogan_2018_fast}.
Treatment of horizontal competition also plays an important role in the outcomes of competition for light between different plants~\citep{fisher2018vegetation}.
A useful avenue for development and parameterization of these models is comparison to more sophisticated and realistic three-dimensional representations of radiative transfer~\citep[e.g.][]{widlowski2007third}, which are themselves too computationally demanding to be coupled to ecosystem models, but from which empirical distributions and response functions could be derived and against which the behavior of simpler models could be evaluated.

% TODO Flesh out
In this study, the vegetation composition at each site (including the PFT distribution and size-age structure) was prescribed in detail based on inventory data.
This allowed us to focus the calibration on model parameters related canopy radiative transfer model parameters.
However, ED2 is a dynamic vegetation model whose core purpose is to predict how vegetation composition and structure evolve through time.
An important future direction of this work is to evaluate such dynamic ED2 simulations where vegetation composition and structure and predicted with some uncertainty.

Cross-validation and out-of-sample validation are useful tests of model performance, and we recommend these activities as future directions for this and similar work.
However, because our calibration was joint across all sites, the marginal benefit of a separate validation at other sites not used in the calibration was relatively low.
With 54 sites in our calibration, any single site represents <2\% of the data, and for a joint calibration without site random effects, we have every reason to believe that the calibration is not overfitting to any individual site.
Trying to fit any one site well would cause others to do worse (especially given the large observed variability in forest structure) unless the EDR model structure was reasonable and the parameters chosen were genuinely good choices.

Modeling conifer canopy reflectance---and estimating vegetation properties from conifer canopies---is challenging for a number of reasons.
At the leaf level, TODO.
At the canopy level, TODO.
However, although some conifer-dominated sites did demonstrate significant biases in reflectance predictions, our analysis of reflectance bias by composition, structure, and spectral region (Figures~\ref{fig:bias-boxplot-pft} and~\ref{fig:bias-density-pft}) shows that these biases are not systematic (though they may drive greater predictive variance).
