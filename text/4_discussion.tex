\section{Discussion}

The accurate simulation of canopy radiative transfer is key to a number of ecosystem processes, including photosynthesis, soil respiration, and hydrology.
For its representation of radiative transfer, ED2 uses a modified version of the two-stream model of \citet{sellers1985canopy} as implemented in the Community Land Model~\citep{clm45_note} but adapted for multiple canopies.

A key feature of EDR design is its representation of multiple co-existing plant cohorts competing for light within a single patch.

A useful avenue for development and parameterization of these models is comparison to more sophisticated and realistic three-dimensional representations of radiative transfer~\citep[e.g.][]{widlowski2007third}, which are themselves too computationally demanding to be coupled to ecosystem models, but from which empirical distributions and response functions could be derived and against which the behavior of simpler models could be evaluated.

% TODO: Do this analysis?
A significant body of remote sensing literature argues that the inversion of coupled leaf-canopy radiative transfer models is ill-posed because of the collinearity of structure and biochemistry effects on canopy reflectance~\citep[e.g.][]{combal2003retrieval, lewis2007spectral}.
Even with tight prior constraint on leaf optical properties from a large meta-analysis, I saw evidence of significant trade-offs between parameters.
In particular, in this analysis, three parameters influenced canopy reflectance in virtually identical ways through modulating the (effective) leaf area index:
specific leaf area, leaf biomass allometry, and clumping factor.
This collinearity resulted in frequent mismatches between observed and modeled leaf area index despite high accuracy in modeled reflectance spectra (Figures~\ref{fig:lai_profile} and~\ref{fig:spec_error_all}).
Additional measurements of poorly constrained but highly influential structural parameters should help alleviate this problem.
Fortunately, these structural metrics are often effectively retrieved from LiDAR observations from terrestrial~\citep{eitel_2016_beyond}, airborne~\citep{antonarakis2014imaging}, and satellite platforms~\citep{coyle_2015_laser}.

Finally, despite many challenges related to canopy radiative transfer modeling, surface reflectance is nevertheless a promising approach for benchmarking and performing data assimilation on ecosystem model outputs.
Remote sensing observations are unrivaled in their spatial completeness and extent, notably extending to regions like the tropics and high latitudes that are relatively undersampled but have a disproportionate impact on the global climate system~\citep{schimel2015observing} and/or global biodiversity~\citep{jetz_2016_monitoring}.
At the same time, satellite time series provide multi-decadal records with relatively high temporal frequency, which have tremendous utility for calibrating model projections of past ecological dynamics~\citep{kennedy_2014_bringing, pasquarella2016imagery}.
Used in combination with other emerging data sources, including global trait databases and eddy covariance measurements, remote sensing can be a transformative force in ecosystem ecology.
