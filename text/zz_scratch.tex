Recognition of the importance of these processes has led to the development of vegetation models with explicit representations of canopy radiative transfer.
The most accurate canopy radiative transfer models capture both vertical and horizontal heterogeneity with very high spatial resolution~\parencite{widlowski2007third}.
However, such models are usually too computationally intensive for dynamic vegetation models, which employ various approximations based on simplifying assumptions to make the problem more tractable~\parencite{fisher2017vegetation}.
One common approach is the ``two-stream approximation'', which simplifies the problem of directional scattering within a medium by modeling the hemispherical integral of fluxes rather than individual, directional components.
In the context of radiative transfer in plant canopies, many different two-stream formulations have been developed, of which I highlight two:
One formulation was developed by Kubelka and Munk (1931)\nocite{kubelka1931article} and later adapted to vegetation canopies by Allen, Gayle, and Richardson (1970)\nocite{allen1970plantcanopy} and further refined by Suits (1971)\nocite{suits1971calculation}, Verhoef (1985)\nocite{verhoef1984light}, and others.
This theory forms the foundation of the SAIL canopy radiative transfer model~\parencite{verhoef1984light} and its derivatives~\parencite[e.g. 4SAIL][]{verhoef2007unified}, which have been used extensively in the remote sensing community for modeling and retrieving vegetation characteristics from spectral data~\cite{jacquemoud2009prospect}.
Another was developed by Meador and Weaver (1980)\nocite{meador1980twostream} for atmospheric radiative transfer, and was subsequently adapted to canopy radiative transfer by Dickinson (1983)\nocite{dickinson_1983_land} and refined by Sellers (1985)\nocite{sellers1985canopy}.
%Key assumptions of this approach are that all diffuse radiative fluxes are isotropic (i.e.\ scattering is equal in all directions) and that all canopy elements are sufficiently far from each other that there is no inter-particle shading~\parencite{meador_1980_twostream,dickinson_1983_land,SELLERS_1985_canopy}.
Due to its theoretical simplicity and low computational demand, this is the approach commonly used to represent radiative transfer in ecosystem models, including the Community Land Model~\parencite[CLM,][]{clm45_note} and the Ecosystem Demography model~\cite[ED,][]{moorcroft_2001_method, medvigy2009mechanistic, longo_2019_ed1}.
The version of this scheme used in ED2 (and derivative models) is fairly unique in its explicit representation of multiple canopy layers, which allows ED2 to simulate competition for light, a key component of modeling vegetation demographics~\parencite{fisher2017vegetation}.
However, compared to physiological processes, the structure and parameterization of canopy radiative transfer schemes in demographic models has received relatively little attention.
When canopy radiative transfer has been considered, it was shown to be important to a wide range of physiological and demographic processes.
For example, using a modified version of the ED model, Fisher et al. (2010)\nocite{fisher_2010_assessing} showed that excessive light absorption by the top cohort resulted in unrealistically excessive growth of canopy trees at the expense of understory trees.
Similarly, an analysis by Viskari et al.\ (in revision) \nocite{viskari_2019_influence} demonstrated that the Ecosystem Demography (ED2) model's predictions of ecosystem energy budget, productivity, and composition are highly sensitive to the parameterization of the model's representation of canopy radiative transfer.
Understanding and improving representations of canopy radiative transfer in dynamic vegetation models is therefore critical to accurate projections of the fate of the terrestrial biosphere.
