\section{Results}

\begin{figure}
  \centering
  \includegraphics[width=\textwidth,height=0.6\textheight,keepaspectratio]{figures/posterior-pft}
  \caption{\label{fig:posterior-pft}\
    Marginal prior (pre-calibration; grey) and posterior (post-calibration; black) distributions of PFT-specific parameters
    related to leaf biochemistry and canopy structure.
    Distributions are shown as violin plots (rotated and mirrored kernel density plots).
    PFTs are abbreviated as follows:
    EH:\@Early Hardwood;
    MH:\@North Mid Hardwood;
    LH:\@Late Hardwood;
    NP:\@Northern Pine;
    LC:\@Late conifer.
    Leaf and wood biomass allometry panels are clipped at 0.2 to facilitate differentiation of posterior distributions.
  }
\end{figure}

Model calibration improved the precision of most PFT-specific parameter estimates, including estimates of leaf parameters whose prior distributions were already independently constrained by an earlier analysis (Figure~\ref{fig:posterior-pft}).
Across all PFT-specific parameters, the posterior 95\% credible interval (CI) was, on average, 10\% the size of the prior credible interval.
The most constrained parameters on average were EDR canopy structure parameters---namely the wood biomass allometry (<1\% of prior CI), leaf biomass allometry (1\%), leaf orientation factor (8\%), and clumping factor (9\%)---
while the least constrained parameters were those related to leaf morphology and biochemistry---namely, effective number of leaf layers (19\%), total chlorphyll content (16\%), total carotenoid content (15\%), specific leaf area (13\%), dry matter content (11\%), and leaf water content (11\%).
By PFT, the largest average relative constraint was for early hardwood (7\%) and the smallest relative constraint was for late hardwood (14\% of prior CI).

For leaf traits, PFT rankings of the posterior estimates largely followed the relative positions of the priors, though there were a few exceptions.
In both the prior and posterior, the estimated effective number of leaf mesophyll layers (a.k.a., PROSPECT \emph{N} parameter) was higher for needleleaved than broadleaved PFTs, with the highest value for northern pine and the lowest value for mid hardwood.
Similarly, specific leaf area (SLA) was lower in conifer than broadleaf PFTs, with the lowest value for late conifer, a higher value for northern pine (despite a similar prior), and higher values still in mid hardwood and late hardwood and the highest value for early hardwood.
Estimated total chlorophyll contents (\emph{Cab}) were similarly high for all hardwood PFTs in both the prior and posterior,
but posterior estimates for late conifer and northern pine were lower.
Posterior estimates of total carotenoid contents (\emph{Car}) were lower in early and mid hardwood and northern pine and higher in mid hardwood and late conifer.
Posterior estimates of leaf water content (\emph{Cw}) were low for early hardwood and northern pine and high for mid- and late hardwood and late conifer; these differences were despite strongly overlapping priors across all PFTs.
Posterior estimates of leaf dry matter content (\emph{Cm}) were lowest for mid- and late hardwood, higher for early hardwood and northern pine, and highest for late conifer, again despite a strongly overlapping prior across all PFTs.

Compared to leaf traits, canopy structural traits had less informative (and PFT-agnostic) priors, and the posterior distributions exhibited some differences across PFTs.
Posterior leaf biomass allometry ($\bbbl$) estimates were lowest in early hardwood and northern pine, higher for late conifer, and highest for late- and mid hardwood.
Posterior wood biomass allometry ($\bbbw$) estimates were lowest for early hardwood and late conifer, slightly higher in mid- and late hardwood, and highest for northern pine.
Posterior canopy clumping factor ($q$) estimates were clustered at or near its upper limit of 1 (i.e., exhibited ``edge-hitting behavior'') for early and mid hardwood, were slightly lower in late hardwood and northern pine, and lowest in late conifer.
Posterior leaf orientation factor ($\chi$) estimates were lowest (near zero, indicating randomly distributed leaf angles) for northern pine, higher (more horizontal leaves) for early and mid hardwoods and late conifer, and highest (at the upper limit of 0.6) for late hardwood.
Finally, the calibration was able to constrain site-specific soil optical properties across all sites (Figure~\ref{fig:posterior-soil}).

\begin{figure}
  \centering
  \includegraphics[width=\textwidth,keepaspectratio]{figures/spec-error-aggregate}
  \caption{\label{fig:spec-error-aggregate}\
    Differences between AVIRIS observed and EDR posterior predictive mean surface reflectance by site.
    Each thin gray line is a site-specific AVIRIS observation.
    Top left panel shows results across all sites,
    and remaining panels group sites according to dominant PFT .
    Within each panel, blue shading shows the 25--75\% quantile range and the thick black line is the median by wavelength for that specific site.
  }
\end{figure}

\begin{figure}
  \centering
  \includegraphics[width=0.7\textwidth]{figures/bias-boxplot-pft}
  \caption{\label{fig:bias-boxplot-pft}\
    Differences between AVIRIS observed and EDR posterior predictive mean surface reflectance by site, averaged across wavelength regions.
    Sites are grouped by dominant PFT, as in Figure~\ref{fig:spec-error-aggregate}.
    Note the differences in the $y$ axis scale across panels.
  }
\end{figure}

\begin{figure}
  \centering
  \includegraphics[width=\textwidth,height=0.8\textheight,keepaspectratio]{figures/mainfig-sites}
  \caption{\label{fig:spec-error-all}\
    (\emph{Left}) Comparison between AVIRIS observed (black) and
    EDR predicted (mean prediction in green, 95\% posterior predictive interval in gray)
    surface reflectance for a geographically (Figure~\ref{fig:site-map}), compositionally, and structurally (Figure~\ref{fig:site-structure}) representative sample of sites used in the calibration.
    (\emph{Right}) Histogram of stem diameter at breast height (DBH) by plant functional type (PFT) at the corresponding site.
  }
\end{figure}

The accuracy and precision of EDR simulated spectra relative to AVIRIS observations varied across sites (Figures~\ref{fig:spec-error-aggregate},~\ref{fig:bias-boxplot-pft},~\ref{fig:spec-error-all}, and~\ref{fig:spec-error-allsites}).
On average, EDR tended to accurately (within 0.01) reproduce reflectance in the 400--750 \unit{nm} range, underpredict AVIRIS reflectance by \~0.03 in the 750--1100 \unit{nm} range, and overpredict AVIRIS reflectance by \~0.04 in the 1100--1300 \unit{nm} range.
However, only the latter behavior was mostly consistent across sites;
below 1100 \unit{nm}, sites of any dominant PFT could have low, accurate, or high estimates relative to AVIRIS.\@
The only consistent biases (expressed as interquartile range in bias not overlapping zero) we observed with respect to PFT were
overestimates of reflectance in the 400--750 \unit{nm} range for late conifer
and
underestimates of reflectance in the 750--1100 \unit{nm} range for northern pine;
otherwise, we did not observe any consistent patterns in mismatch between AVIRIS observed and EDR predicted reflectance with respect to tree size, stem density, or composition (Figures~\ref{fig:bias-boxplot-pft} and~\ref{fig:bias-density-pft}--\ref{fig:tree-sites-LC}).
The EDR posterior predictive interval overlapped AVIRIS observations in all but a few individual cases (Figure~\ref{fig:spec-error-allsites}), suggesting that our estimates of model uncertainty are realistic.

\begin{figure}
  \centering
  \includegraphics[width=3in]{figures/lai-pred-obs}
  \caption{\
    EDR predictions of site-specific true leaf area index (LAI) compared to observed values.
    Horizontal error bars are posterior 95\% predictive intervals.
    Vertical error bars are mean $\pm$ 1 standard deviation of the observed values.
    Dashed line shows the 1:1 relationship, and solid line is a least-squares predicted vs.\ observed regression with the equation marked in the upper left corner.
    Points are colored according to dominant PFT, calculated as in Figure~\ref{fig:spec-error-aggregate}.
  }\label{fig:lai-pred-obs}
\end{figure}

\begin{figure}
  \centering
  \includegraphics[width=0.75\textwidth]{figures/lai-bias-dbh-bypft}
  \caption{\
    Bias in leaf area index (LAI) predictions from calibrated EDR relative to observations,
    as a function of site mean diameter at breast height (DBH).
    Sites are grouped according to dominant PFT, same as in Figure~\ref{fig:spec-error-aggregate}.
  }\label{fig:lai-bias-dbh-bypft}
\end{figure}

\begin{figure}
  \centering
  \includegraphics[width=0.75\textwidth]{figures/lai-bias-dens-bypft}
  \caption{\
    Same as above, but as a function of mean site stem density.
  }\label{fig:lai-bias-dens-bypft}
\end{figure}

Leaf area index predicted from calibrated EDR parameters captured 53\% of the variability in the observations (Figure~\ref{fig:lai-pred-obs}).
The observed vs.\ predicted line had a slope of 0.40 and an intercept of 2.82, indicating that EDR calibration underpredicted true LAI on average but exaggerated across-site LAI variability.
In general, EDR tended to underpredict LAI at high-density sites with low mean DBH and overpredict LAI at low-density sites with high mean DBH (Figures~\ref{fig:lai-bias-dbh-bypft} and~\ref{fig:lai-bias-dens-bypft}).
The trend with mean DBH was generally true across all PFTs but was most pronounced for early hardwood- and late conifer-dominated sites (Figure~\ref{fig:lai-bias-dbh-bypft}),
while the trend with stand density was most pronounced for late conifer-dominated sites (Figure~\ref{fig:lai-bias-dens-bypft}).

\begin{figure}
  \centering
  \includegraphics[width=0.7\textwidth]{figures/edr-sail-comparison-czen}
  \caption{\label{fig:edr-sail-comparison-czen}\
    Comparison of EDR and PRO4SAIL (labelled as ``SAIL'' for conciseness) predictions of reflectance for identical, single-cohort canopies as a function of solar zenith angle ($\theta_{s}$).
    These simulations use identical PROSPECT and canopy structure parameters, a nadir-viewing sensor, and LAI of 3.
    For EDR, we use a single cohort with LAI of 3 prescribed directly.\@
    ``SAIL:\@HR'' is the ``hemispherical reflectance'' (``blue-sky albedo''), calculated (as in EDR) as the average of directional-hemispherical (DHR) and bi-hemispherical (BHR) reflectance streams weighted by the direct sky fraction (0.9; same value used for EDR).
    Similarly, ``SAIL:\@DR'' is the ``directional reflectance'', calculated analogously from the bi-directional (BDR) and hemispherical-directional (HDR) reflectance streams.
    Individual SAIL fluxes are shown with dotted lines.
  }
\end{figure}

For identical canopies, EDR consistently predicted lower hemispherical reflectance than PRO4SAIL (Figure~\ref{fig:edr-sail-comparison-czen}).
This difference was most pronounced when the sun was directly overhead ($\theta_{s} = 0\degree$; $\cos(\theta_{s}) = 1$) and declined with increasing solar zenith angle.
For solar zenith angles typical of our study, ($\theta_{s} \approx 30\degree$; $\cos(\theta_{s}) = 0.85$) EDR hemispherical reflectance predictions were very close to PRO4SAIL directional reflectance predictions over a wide range of LAI values (Figure~\ref{fig:edr-sail-comparison-lai}).
