\section{Supplementary figures}

\begin{figure}[ht]
  \centering
  \includegraphics[width=3in]{figures/edr-sensitivity-lai}
  \caption{\label{fig:edr-sensitivity-lai}\
    Sensitivity of EDR predicted hemispherical reflectance to true leaf area index (LAI).
    These simulations assume a single-cohort canopy with
    effective number of mesophyll layers $N = 1.4$,
    total chlorophyll content $Cab = 40$,
    total carotenoid content $Car = 10$,
    leaf water content $Cw = 0.01$,
    leaf dry matter content $Cm = 0.01$,
    clumping factor $q = 1$,
    leaf orientation factor $\chi = 0$,
    $\cos(\theta_{s}) = 0.85$,
    and soil moisture fraction $\psi = 0.5$.
  }
\end{figure}

\clearpage

\begin{figure}[ht]
  \centering
  \includegraphics[width=3in]{figures/edr-sensitivity-clumping_factor}
  \caption{\label{fig:edr-sensitivity-clumping}\
    Same as above but with LAI fixed to 3 and varying clumping factor ($q$).
  }
\end{figure}

\clearpage

\begin{figure}[ht]
  \centering
  \includegraphics[width=3in]{figures/edr-sensitivity-orient_factor}
  \caption{\label{fig:edr-sensitivity-orient}\
    Same as above, but instead varying leaf orientation factor ($\chi$).
  }
\end{figure}

\clearpage

\begin{figure}[ht]
  \centering
  \includegraphics[width=\textwidth]{figures/posterior-soil}
  \caption{\label{fig:posterior-soil}\
    Site-specific relative soil moisture (0 = dry, 1 = wet) posterior estimates.
    Sites are sorted in order of increasing weighted evergreen fraction.
  }
\end{figure}

\clearpage

\begin{figure}
  \centering
  \includegraphics[width=\textwidth,height=0.8\textheight,keepaspectratio]{figures/spec-error-all}
  \caption{\label{fig:spec-error-allsites}\
    Comparison between AVIRIS observed (black) and
    surface reflectance for each site used in the calibration.
    Sites are sorted in order of decreasing mean difference between observed and EDR predicted reflectance
    (largest underestimates first, largest overestimates last).
  }
\end{figure}

\clearpage

\begin{figure}
  \centering
  \includegraphics[width=\textwidth]{figures/bias-density-pft}
  \caption{\label{fig:bias-density-pft}\
    Mean reflectance bias (EDR predicted $-$ observed) for each by spectral region and dominant plant functional type (PFT) as a function of site stem density.
    PFTs are abbreviated as follows:
    EH:\@Early Hardwood;
    MH:\@North Mid Hardwood;
    LH:\@Late Hardwood;
    NP:\@Northern Pine;
    LC:\@Late conifer
  }
\end{figure}

\clearpage

\begin{figure}[ht]
  \centering
  \includegraphics[width=\textwidth,height=0.8\textheight,keepaspectratio]{figures/tree-sites-q0}
  \caption{\label{fig:tree-sites-q0}\
    EDR predicted vs.\ observed spectra and species composition for the first quartile of sites by DBH.\@
  }
\end{figure}

\clearpage

\begin{figure}[ht]
  \centering
  \includegraphics[width=\textwidth,height=0.8\textheight,keepaspectratio]{figures/tree-sites-q25}
  \caption{\label{fig:tree-sites-q25}\
    As above, but for the second quartile of sites by DBH.\@
  }
\end{figure}

\clearpage

\begin{figure}[ht]
  \centering
  \includegraphics[width=\textwidth,height=0.8\textheight,keepaspectratio]{figures/tree-sites-q50}
  \caption{\label{fig:tree-sites-q50}\
    As above, but for the third quartile of sites by DBH.\@
  }
\end{figure}

\clearpage

\begin{figure}[ht]
  \centering
  \includegraphics[width=\textwidth,height=0.8\textheight,keepaspectratio]{figures/tree-sites-q75}
  \caption{\label{fig:tree-sites-q75}\
    As above, but for the fourth quartile of sites by DBH.\@
  }
\end{figure}

\clearpage

\begin{figure}[ht]
  \centering
  \includegraphics[width=\textwidth,height=0.8\textheight,keepaspectratio]{figures/tree-sites-Early_Hardwood}
  \caption{\label{fig:tree-sites-EH}\
    As above, but for sites where Early Hardwood trees had the largest mean DBH.\@
  }
\end{figure}

\clearpage

\begin{figure}[ht]
  \centering
  \includegraphics[width=\textwidth,height=0.8\textheight,keepaspectratio]{figures/tree-sites-North_Mid_Hardwood}
  \caption{\label{fig:tree-sites-MH}\
    As above, but for sites where Mid Hardwood trees had the largest mean DBH.\@
  }
\end{figure}

\clearpage

\begin{figure}[ht]
  \centering
  \includegraphics[width=\textwidth,height=0.8\textheight,keepaspectratio]{figures/tree-sites-Late_Hardwood}
  \caption{\label{fig:tree-sites-LH}\
    As above, but for sites where Late Hardwood trees had the largest mean DBH.\@
  }
\end{figure}

\clearpage

\begin{figure}[ht]
  \centering
  \includegraphics[width=\textwidth,height=0.8\textheight,keepaspectratio]{figures/tree-sites-Northern_Pine}
  \caption{\label{fig:tree-sites-P}\
    As above, but for sites where Pine trees had the largest mean DBH.\@
  }
\end{figure}

\clearpage

\begin{figure}[ht]
  \centering
  \includegraphics[width=\textwidth,height=0.8\textheight,keepaspectratio]{figures/tree-sites-Late_Conifer}
  \caption{\label{fig:tree-sites-LC}\
    As above, but for sites where Late Conifer trees had the largest mean DBH.\@
  }
\end{figure}

\clearpage

\begin{figure}[ht]
  \centering
  \includegraphics[width=\textwidth]{figures/edr-sail-comparison-lai}
  \caption{\label{fig:edr-sail-comparison-lai}\
    Same as Figure~\ref{fig:edr-sail-comparison-czen}, but varying leaf area index (LAI) and fixing $\cos(\theta_{s}) = 0.85$, a typical value for our study.
  }
\end{figure}

\clearpage

\begin{figure}[ht]
  \centering
  \includegraphics[width=\textwidth]{figures/posterior-correlations}
  \caption{\label{fig:posterior-correlations}\
    Posterior correlation matrix for PFT-specific parameters.
    Note that only correlations among parameters within the same PFT are shown---the full 106 $\times$ 106 dimensional correlation matrix is far too large to display in this format.
  }
\end{figure}
