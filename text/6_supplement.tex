\section{Supplementary information}

\begin{figure}[ht]
  \centering
  \includegraphics[width=\textwidth]{figures/posterior-soil}
  \caption{\label{fig:posterior-soil}\
    Site-specific relative soil moisture (0 = dry, 1 = wet) posterior estimates.
    Sites are sorted in order of increasing weighted evergreen fraction.
  }
\end{figure}

\clearpage

\begin{figure}[ht]
  \centering
  \includegraphics[width=\textwidth]{figures/ndvi-dbh}
  \caption{\label{fig:ndvi-dbh}\
    Predicted and observed site normalized difference vegetation index (NDVI, with Red = 690 nm and NIR = 800 nm)
    as a function of site mean diameter at breast height (DBH), with linear regression.
  }
\end{figure}

\clearpage

\begin{figure}[ht]
  \centering
  \includegraphics[width=\textwidth,height=0.8*\textheight,keepaspectratio]{figures/tree-sites-q0}
  \caption{\label{fig:tree-sites-q0}\
    EDR predicted vs.\ observed spectra and species composition for the first quartile of sites by DBH.
  }
\end{figure}

\clearpage

\begin{figure}[ht]
  \centering
  \includegraphics[width=\textwidth,height=0.8*\textheight,keepaspectratio]{figures/tree-sites-q25}
  \caption{\label{fig:tree-sites-q25}\
    As above, but for the second quartile of sites by DBH.
  }
\end{figure}

\clearpage

\begin{figure}[ht]
  \centering
  \includegraphics[width=\textwidth,height=0.8*\textheight,keepaspectratio]{figures/tree-sites-q50}
  \caption{\label{fig:tree-sites-q50}\
    As above, but for the third quartile of sites by DBH.
  }
\end{figure}

\clearpage

\begin{figure}[ht]
  \centering
  \includegraphics[width=\textwidth,height=0.8*\textheight,keepaspectratio]{figures/tree-sites-q75}
  \caption{\label{fig:tree-sites-q75}\
    As above, but for the fourth quartile of sites by DBH.
  }
\end{figure}

\clearpage

\begin{figure}[ht]
  \centering
  \includegraphics[width=\textwidth,height=0.8*\textheight,keepaspectratio]{figures/tree-sites-Early_Hardwood}
  \caption{\label{fig:tree-sites-EH}\
    As above, but for sites where Early Hardwood trees had the largest mean DBH.
  }
\end{figure}

\clearpage

\begin{figure}[ht]
  \centering
  \includegraphics[width=\textwidth,height=0.8*\textheight,keepaspectratio]{figures/tree-sites-North_Mid_Hardwood}
  \caption{\label{fig:tree-sites-MH}\
    As above, but for sites where Mid Hardwood trees had the largest mean DBH.
  }
\end{figure}

\clearpage

\begin{figure}[ht]
  \centering
  \includegraphics[width=\textwidth,height=0.8*\textheight,keepaspectratio]{figures/tree-sites-Late_Hardwood}
  \caption{\label{fig:tree-sites-LH}\
    As above, but for sites where Late Hardwood trees had the largest mean DBH.
  }
\end{figure}

\clearpage

\begin{figure}[ht]
  \centering
  \includegraphics[width=\textwidth,height=0.8*\textheight,keepaspectratio]{figures/tree-sites-Northern_Pine}
  \caption{\label{fig:tree-sites-P}\
    As above, but for sites where Pine trees had the largest mean DBH.
  }
\end{figure}

\clearpage

\begin{figure}[ht]
  \centering
  \includegraphics[width=\textwidth,height=0.8*\textheight,keepaspectratio]{figures/tree-sites-Late_Conifer}
  \caption{\label{fig:tree-sites-LC}\
    As above, but for sites where Late Conifer trees had the largest mean DBH.
  }
\end{figure}
